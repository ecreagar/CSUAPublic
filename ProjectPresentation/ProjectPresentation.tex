\documentclass[11pt]{article}

    \usepackage[breakable]{tcolorbox}
    \usepackage{parskip} % Stop auto-indenting (to mimic markdown behaviour)
    
    \usepackage{iftex}
    \ifPDFTeX
    	\usepackage[T1]{fontenc}
    	\usepackage{mathpazo}
    \else
    	\usepackage{fontspec}
    \fi

    % Basic figure setup, for now with no caption control since it's done
    % automatically by Pandoc (which extracts ![](path) syntax from Markdown).
    \usepackage{graphicx}
    % Maintain compatibility with old templates. Remove in nbconvert 6.0
    \let\Oldincludegraphics\includegraphics
    % Ensure that by default, figures have no caption (until we provide a
    % proper Figure object with a Caption API and a way to capture that
    % in the conversion process - todo).
    \usepackage{caption}
    \DeclareCaptionFormat{nocaption}{}
    \captionsetup{format=nocaption,aboveskip=0pt,belowskip=0pt}

    \usepackage[Export]{adjustbox} % Used to constrain images to a maximum size
    \adjustboxset{max size={0.9\linewidth}{0.9\paperheight}}
    \usepackage{float}
    \floatplacement{figure}{H} % forces figures to be placed at the correct location
    \usepackage{xcolor} % Allow colors to be defined
    \usepackage{enumerate} % Needed for markdown enumerations to work
    \usepackage{geometry} % Used to adjust the document margins
    \usepackage{amsmath} % Equations
    \usepackage{amssymb} % Equations
    \usepackage{textcomp} % defines textquotesingle
    % Hack from http://tex.stackexchange.com/a/47451/13684:
    \AtBeginDocument{%
        \def\PYZsq{\textquotesingle}% Upright quotes in Pygmentized code
    }
    \usepackage{upquote} % Upright quotes for verbatim code
    \usepackage{eurosym} % defines \euro
    \usepackage[mathletters]{ucs} % Extended unicode (utf-8) support
    \usepackage{fancyvrb} % verbatim replacement that allows latex
    \usepackage{grffile} % extends the file name processing of package graphics 
                         % to support a larger range
    \makeatletter % fix for grffile with XeLaTeX
    \def\Gread@@xetex#1{%
      \IfFileExists{"\Gin@base".bb}%
      {\Gread@eps{\Gin@base.bb}}%
      {\Gread@@xetex@aux#1}%
    }
    \makeatother

    % The hyperref package gives us a pdf with properly built
    % internal navigation ('pdf bookmarks' for the table of contents,
    % internal cross-reference links, web links for URLs, etc.)
    \usepackage{hyperref}
    % The default LaTeX title has an obnoxious amount of whitespace. By default,
    % titling removes some of it. It also provides customization options.
    \usepackage{titling}
    \usepackage{longtable} % longtable support required by pandoc >1.10
    \usepackage{booktabs}  % table support for pandoc > 1.12.2
    \usepackage[inline]{enumitem} % IRkernel/repr support (it uses the enumerate* environment)
    \usepackage[normalem]{ulem} % ulem is needed to support strikethroughs (\sout)
                                % normalem makes italics be italics, not underlines
    \usepackage{mathrsfs}
    

    
    % Colors for the hyperref package
    \definecolor{urlcolor}{rgb}{0,.145,.698}
    \definecolor{linkcolor}{rgb}{.71,0.21,0.01}
    \definecolor{citecolor}{rgb}{.12,.54,.11}

    % ANSI colors
    \definecolor{ansi-black}{HTML}{3E424D}
    \definecolor{ansi-black-intense}{HTML}{282C36}
    \definecolor{ansi-red}{HTML}{E75C58}
    \definecolor{ansi-red-intense}{HTML}{B22B31}
    \definecolor{ansi-green}{HTML}{00A250}
    \definecolor{ansi-green-intense}{HTML}{007427}
    \definecolor{ansi-yellow}{HTML}{DDB62B}
    \definecolor{ansi-yellow-intense}{HTML}{B27D12}
    \definecolor{ansi-blue}{HTML}{208FFB}
    \definecolor{ansi-blue-intense}{HTML}{0065CA}
    \definecolor{ansi-magenta}{HTML}{D160C4}
    \definecolor{ansi-magenta-intense}{HTML}{A03196}
    \definecolor{ansi-cyan}{HTML}{60C6C8}
    \definecolor{ansi-cyan-intense}{HTML}{258F8F}
    \definecolor{ansi-white}{HTML}{C5C1B4}
    \definecolor{ansi-white-intense}{HTML}{A1A6B2}
    \definecolor{ansi-default-inverse-fg}{HTML}{FFFFFF}
    \definecolor{ansi-default-inverse-bg}{HTML}{000000}

    % commands and environments needed by pandoc snippets
    % extracted from the output of `pandoc -s`
    \providecommand{\tightlist}{%
      \setlength{\itemsep}{0pt}\setlength{\parskip}{0pt}}
    \DefineVerbatimEnvironment{Highlighting}{Verbatim}{commandchars=\\\{\}}
    % Add ',fontsize=\small' for more characters per line
    \newenvironment{Shaded}{}{}
    \newcommand{\KeywordTok}[1]{\textcolor[rgb]{0.00,0.44,0.13}{\textbf{{#1}}}}
    \newcommand{\DataTypeTok}[1]{\textcolor[rgb]{0.56,0.13,0.00}{{#1}}}
    \newcommand{\DecValTok}[1]{\textcolor[rgb]{0.25,0.63,0.44}{{#1}}}
    \newcommand{\BaseNTok}[1]{\textcolor[rgb]{0.25,0.63,0.44}{{#1}}}
    \newcommand{\FloatTok}[1]{\textcolor[rgb]{0.25,0.63,0.44}{{#1}}}
    \newcommand{\CharTok}[1]{\textcolor[rgb]{0.25,0.44,0.63}{{#1}}}
    \newcommand{\StringTok}[1]{\textcolor[rgb]{0.25,0.44,0.63}{{#1}}}
    \newcommand{\CommentTok}[1]{\textcolor[rgb]{0.38,0.63,0.69}{\textit{{#1}}}}
    \newcommand{\OtherTok}[1]{\textcolor[rgb]{0.00,0.44,0.13}{{#1}}}
    \newcommand{\AlertTok}[1]{\textcolor[rgb]{1.00,0.00,0.00}{\textbf{{#1}}}}
    \newcommand{\FunctionTok}[1]{\textcolor[rgb]{0.02,0.16,0.49}{{#1}}}
    \newcommand{\RegionMarkerTok}[1]{{#1}}
    \newcommand{\ErrorTok}[1]{\textcolor[rgb]{1.00,0.00,0.00}{\textbf{{#1}}}}
    \newcommand{\NormalTok}[1]{{#1}}
    
    % Additional commands for more recent versions of Pandoc
    \newcommand{\ConstantTok}[1]{\textcolor[rgb]{0.53,0.00,0.00}{{#1}}}
    \newcommand{\SpecialCharTok}[1]{\textcolor[rgb]{0.25,0.44,0.63}{{#1}}}
    \newcommand{\VerbatimStringTok}[1]{\textcolor[rgb]{0.25,0.44,0.63}{{#1}}}
    \newcommand{\SpecialStringTok}[1]{\textcolor[rgb]{0.73,0.40,0.53}{{#1}}}
    \newcommand{\ImportTok}[1]{{#1}}
    \newcommand{\DocumentationTok}[1]{\textcolor[rgb]{0.73,0.13,0.13}{\textit{{#1}}}}
    \newcommand{\AnnotationTok}[1]{\textcolor[rgb]{0.38,0.63,0.69}{\textbf{\textit{{#1}}}}}
    \newcommand{\CommentVarTok}[1]{\textcolor[rgb]{0.38,0.63,0.69}{\textbf{\textit{{#1}}}}}
    \newcommand{\VariableTok}[1]{\textcolor[rgb]{0.10,0.09,0.49}{{#1}}}
    \newcommand{\ControlFlowTok}[1]{\textcolor[rgb]{0.00,0.44,0.13}{\textbf{{#1}}}}
    \newcommand{\OperatorTok}[1]{\textcolor[rgb]{0.40,0.40,0.40}{{#1}}}
    \newcommand{\BuiltInTok}[1]{{#1}}
    \newcommand{\ExtensionTok}[1]{{#1}}
    \newcommand{\PreprocessorTok}[1]{\textcolor[rgb]{0.74,0.48,0.00}{{#1}}}
    \newcommand{\AttributeTok}[1]{\textcolor[rgb]{0.49,0.56,0.16}{{#1}}}
    \newcommand{\InformationTok}[1]{\textcolor[rgb]{0.38,0.63,0.69}{\textbf{\textit{{#1}}}}}
    \newcommand{\WarningTok}[1]{\textcolor[rgb]{0.38,0.63,0.69}{\textbf{\textit{{#1}}}}}
    
    
    % Define a nice break command that doesn't care if a line doesn't already
    % exist.
    \def\br{\hspace*{\fill} \\* }
    % Math Jax compatibility definitions
    \def\gt{>}
    \def\lt{<}
    \let\Oldtex\TeX
    \let\Oldlatex\LaTeX
    \renewcommand{\TeX}{\textrm{\Oldtex}}
    \renewcommand{\LaTeX}{\textrm{\Oldlatex}}
    % Document parameters
    % Document title
    \title{ProjectPresentation}
    
   
    
    
% Pygments definitions
\makeatletter
\def\PY@reset{\let\PY@it=\relax \let\PY@bf=\relax%
    \let\PY@ul=\relax \let\PY@tc=\relax%
    \let\PY@bc=\relax \let\PY@ff=\relax}
\def\PY@tok#1{\csname PY@tok@#1\endcsname}
\def\PY@toks#1+{\ifx\relax#1\empty\else%
    \PY@tok{#1}\expandafter\PY@toks\fi}
\def\PY@do#1{\PY@bc{\PY@tc{\PY@ul{%
    \PY@it{\PY@bf{\PY@ff{#1}}}}}}}
\def\PY#1#2{\PY@reset\PY@toks#1+\relax+\PY@do{#2}}

\expandafter\def\csname PY@tok@w\endcsname{\def\PY@tc##1{\textcolor[rgb]{0.73,0.73,0.73}{##1}}}
\expandafter\def\csname PY@tok@c\endcsname{\let\PY@it=\textit\def\PY@tc##1{\textcolor[rgb]{0.25,0.50,0.50}{##1}}}
\expandafter\def\csname PY@tok@cp\endcsname{\def\PY@tc##1{\textcolor[rgb]{0.74,0.48,0.00}{##1}}}
\expandafter\def\csname PY@tok@k\endcsname{\let\PY@bf=\textbf\def\PY@tc##1{\textcolor[rgb]{0.00,0.50,0.00}{##1}}}
\expandafter\def\csname PY@tok@kp\endcsname{\def\PY@tc##1{\textcolor[rgb]{0.00,0.50,0.00}{##1}}}
\expandafter\def\csname PY@tok@kt\endcsname{\def\PY@tc##1{\textcolor[rgb]{0.69,0.00,0.25}{##1}}}
\expandafter\def\csname PY@tok@o\endcsname{\def\PY@tc##1{\textcolor[rgb]{0.40,0.40,0.40}{##1}}}
\expandafter\def\csname PY@tok@ow\endcsname{\let\PY@bf=\textbf\def\PY@tc##1{\textcolor[rgb]{0.67,0.13,1.00}{##1}}}
\expandafter\def\csname PY@tok@nb\endcsname{\def\PY@tc##1{\textcolor[rgb]{0.00,0.50,0.00}{##1}}}
\expandafter\def\csname PY@tok@nf\endcsname{\def\PY@tc##1{\textcolor[rgb]{0.00,0.00,1.00}{##1}}}
\expandafter\def\csname PY@tok@nc\endcsname{\let\PY@bf=\textbf\def\PY@tc##1{\textcolor[rgb]{0.00,0.00,1.00}{##1}}}
\expandafter\def\csname PY@tok@nn\endcsname{\let\PY@bf=\textbf\def\PY@tc##1{\textcolor[rgb]{0.00,0.00,1.00}{##1}}}
\expandafter\def\csname PY@tok@ne\endcsname{\let\PY@bf=\textbf\def\PY@tc##1{\textcolor[rgb]{0.82,0.25,0.23}{##1}}}
\expandafter\def\csname PY@tok@nv\endcsname{\def\PY@tc##1{\textcolor[rgb]{0.10,0.09,0.49}{##1}}}
\expandafter\def\csname PY@tok@no\endcsname{\def\PY@tc##1{\textcolor[rgb]{0.53,0.00,0.00}{##1}}}
\expandafter\def\csname PY@tok@nl\endcsname{\def\PY@tc##1{\textcolor[rgb]{0.63,0.63,0.00}{##1}}}
\expandafter\def\csname PY@tok@ni\endcsname{\let\PY@bf=\textbf\def\PY@tc##1{\textcolor[rgb]{0.60,0.60,0.60}{##1}}}
\expandafter\def\csname PY@tok@na\endcsname{\def\PY@tc##1{\textcolor[rgb]{0.49,0.56,0.16}{##1}}}
\expandafter\def\csname PY@tok@nt\endcsname{\let\PY@bf=\textbf\def\PY@tc##1{\textcolor[rgb]{0.00,0.50,0.00}{##1}}}
\expandafter\def\csname PY@tok@nd\endcsname{\def\PY@tc##1{\textcolor[rgb]{0.67,0.13,1.00}{##1}}}
\expandafter\def\csname PY@tok@s\endcsname{\def\PY@tc##1{\textcolor[rgb]{0.73,0.13,0.13}{##1}}}
\expandafter\def\csname PY@tok@sd\endcsname{\let\PY@it=\textit\def\PY@tc##1{\textcolor[rgb]{0.73,0.13,0.13}{##1}}}
\expandafter\def\csname PY@tok@si\endcsname{\let\PY@bf=\textbf\def\PY@tc##1{\textcolor[rgb]{0.73,0.40,0.53}{##1}}}
\expandafter\def\csname PY@tok@se\endcsname{\let\PY@bf=\textbf\def\PY@tc##1{\textcolor[rgb]{0.73,0.40,0.13}{##1}}}
\expandafter\def\csname PY@tok@sr\endcsname{\def\PY@tc##1{\textcolor[rgb]{0.73,0.40,0.53}{##1}}}
\expandafter\def\csname PY@tok@ss\endcsname{\def\PY@tc##1{\textcolor[rgb]{0.10,0.09,0.49}{##1}}}
\expandafter\def\csname PY@tok@sx\endcsname{\def\PY@tc##1{\textcolor[rgb]{0.00,0.50,0.00}{##1}}}
\expandafter\def\csname PY@tok@m\endcsname{\def\PY@tc##1{\textcolor[rgb]{0.40,0.40,0.40}{##1}}}
\expandafter\def\csname PY@tok@gh\endcsname{\let\PY@bf=\textbf\def\PY@tc##1{\textcolor[rgb]{0.00,0.00,0.50}{##1}}}
\expandafter\def\csname PY@tok@gu\endcsname{\let\PY@bf=\textbf\def\PY@tc##1{\textcolor[rgb]{0.50,0.00,0.50}{##1}}}
\expandafter\def\csname PY@tok@gd\endcsname{\def\PY@tc##1{\textcolor[rgb]{0.63,0.00,0.00}{##1}}}
\expandafter\def\csname PY@tok@gi\endcsname{\def\PY@tc##1{\textcolor[rgb]{0.00,0.63,0.00}{##1}}}
\expandafter\def\csname PY@tok@gr\endcsname{\def\PY@tc##1{\textcolor[rgb]{1.00,0.00,0.00}{##1}}}
\expandafter\def\csname PY@tok@ge\endcsname{\let\PY@it=\textit}
\expandafter\def\csname PY@tok@gs\endcsname{\let\PY@bf=\textbf}
\expandafter\def\csname PY@tok@gp\endcsname{\let\PY@bf=\textbf\def\PY@tc##1{\textcolor[rgb]{0.00,0.00,0.50}{##1}}}
\expandafter\def\csname PY@tok@go\endcsname{\def\PY@tc##1{\textcolor[rgb]{0.53,0.53,0.53}{##1}}}
\expandafter\def\csname PY@tok@gt\endcsname{\def\PY@tc##1{\textcolor[rgb]{0.00,0.27,0.87}{##1}}}
\expandafter\def\csname PY@tok@err\endcsname{\def\PY@bc##1{\setlength{\fboxsep}{0pt}\fcolorbox[rgb]{1.00,0.00,0.00}{1,1,1}{\strut ##1}}}
\expandafter\def\csname PY@tok@kc\endcsname{\let\PY@bf=\textbf\def\PY@tc##1{\textcolor[rgb]{0.00,0.50,0.00}{##1}}}
\expandafter\def\csname PY@tok@kd\endcsname{\let\PY@bf=\textbf\def\PY@tc##1{\textcolor[rgb]{0.00,0.50,0.00}{##1}}}
\expandafter\def\csname PY@tok@kn\endcsname{\let\PY@bf=\textbf\def\PY@tc##1{\textcolor[rgb]{0.00,0.50,0.00}{##1}}}
\expandafter\def\csname PY@tok@kr\endcsname{\let\PY@bf=\textbf\def\PY@tc##1{\textcolor[rgb]{0.00,0.50,0.00}{##1}}}
\expandafter\def\csname PY@tok@bp\endcsname{\def\PY@tc##1{\textcolor[rgb]{0.00,0.50,0.00}{##1}}}
\expandafter\def\csname PY@tok@fm\endcsname{\def\PY@tc##1{\textcolor[rgb]{0.00,0.00,1.00}{##1}}}
\expandafter\def\csname PY@tok@vc\endcsname{\def\PY@tc##1{\textcolor[rgb]{0.10,0.09,0.49}{##1}}}
\expandafter\def\csname PY@tok@vg\endcsname{\def\PY@tc##1{\textcolor[rgb]{0.10,0.09,0.49}{##1}}}
\expandafter\def\csname PY@tok@vi\endcsname{\def\PY@tc##1{\textcolor[rgb]{0.10,0.09,0.49}{##1}}}
\expandafter\def\csname PY@tok@vm\endcsname{\def\PY@tc##1{\textcolor[rgb]{0.10,0.09,0.49}{##1}}}
\expandafter\def\csname PY@tok@sa\endcsname{\def\PY@tc##1{\textcolor[rgb]{0.73,0.13,0.13}{##1}}}
\expandafter\def\csname PY@tok@sb\endcsname{\def\PY@tc##1{\textcolor[rgb]{0.73,0.13,0.13}{##1}}}
\expandafter\def\csname PY@tok@sc\endcsname{\def\PY@tc##1{\textcolor[rgb]{0.73,0.13,0.13}{##1}}}
\expandafter\def\csname PY@tok@dl\endcsname{\def\PY@tc##1{\textcolor[rgb]{0.73,0.13,0.13}{##1}}}
\expandafter\def\csname PY@tok@s2\endcsname{\def\PY@tc##1{\textcolor[rgb]{0.73,0.13,0.13}{##1}}}
\expandafter\def\csname PY@tok@sh\endcsname{\def\PY@tc##1{\textcolor[rgb]{0.73,0.13,0.13}{##1}}}
\expandafter\def\csname PY@tok@s1\endcsname{\def\PY@tc##1{\textcolor[rgb]{0.73,0.13,0.13}{##1}}}
\expandafter\def\csname PY@tok@mb\endcsname{\def\PY@tc##1{\textcolor[rgb]{0.40,0.40,0.40}{##1}}}
\expandafter\def\csname PY@tok@mf\endcsname{\def\PY@tc##1{\textcolor[rgb]{0.40,0.40,0.40}{##1}}}
\expandafter\def\csname PY@tok@mh\endcsname{\def\PY@tc##1{\textcolor[rgb]{0.40,0.40,0.40}{##1}}}
\expandafter\def\csname PY@tok@mi\endcsname{\def\PY@tc##1{\textcolor[rgb]{0.40,0.40,0.40}{##1}}}
\expandafter\def\csname PY@tok@il\endcsname{\def\PY@tc##1{\textcolor[rgb]{0.40,0.40,0.40}{##1}}}
\expandafter\def\csname PY@tok@mo\endcsname{\def\PY@tc##1{\textcolor[rgb]{0.40,0.40,0.40}{##1}}}
\expandafter\def\csname PY@tok@ch\endcsname{\let\PY@it=\textit\def\PY@tc##1{\textcolor[rgb]{0.25,0.50,0.50}{##1}}}
\expandafter\def\csname PY@tok@cm\endcsname{\let\PY@it=\textit\def\PY@tc##1{\textcolor[rgb]{0.25,0.50,0.50}{##1}}}
\expandafter\def\csname PY@tok@cpf\endcsname{\let\PY@it=\textit\def\PY@tc##1{\textcolor[rgb]{0.25,0.50,0.50}{##1}}}
\expandafter\def\csname PY@tok@c1\endcsname{\let\PY@it=\textit\def\PY@tc##1{\textcolor[rgb]{0.25,0.50,0.50}{##1}}}
\expandafter\def\csname PY@tok@cs\endcsname{\let\PY@it=\textit\def\PY@tc##1{\textcolor[rgb]{0.25,0.50,0.50}{##1}}}

\def\PYZbs{\char`\\}
\def\PYZus{\char`\_}
\def\PYZob{\char`\{}
\def\PYZcb{\char`\}}
\def\PYZca{\char`\^}
\def\PYZam{\char`\&}
\def\PYZlt{\char`\<}
\def\PYZgt{\char`\>}
\def\PYZsh{\char`\#}
\def\PYZpc{\char`\%}
\def\PYZdl{\char`\$}
\def\PYZhy{\char`\-}
\def\PYZsq{\char`\'}
\def\PYZdq{\char`\"}
\def\PYZti{\char`\~}
% for compatibility with earlier versions
\def\PYZat{@}
\def\PYZlb{[}
\def\PYZrb{]}
\makeatother


    % For linebreaks inside Verbatim environment from package fancyvrb. 
    \makeatletter
        \newbox\Wrappedcontinuationbox 
        \newbox\Wrappedvisiblespacebox 
        \newcommand*\Wrappedvisiblespace {\textcolor{red}{\textvisiblespace}} 
        \newcommand*\Wrappedcontinuationsymbol {\textcolor{red}{\llap{\tiny$\m@th\hookrightarrow$}}} 
        \newcommand*\Wrappedcontinuationindent {3ex } 
        \newcommand*\Wrappedafterbreak {\kern\Wrappedcontinuationindent\copy\Wrappedcontinuationbox} 
        % Take advantage of the already applied Pygments mark-up to insert 
        % potential linebreaks for TeX processing. 
        %        {, <, #, %, $, ' and ": go to next line. 
        %        _, }, ^, &, >, - and ~: stay at end of broken line. 
        % Use of \textquotesingle for straight quote. 
        \newcommand*\Wrappedbreaksatspecials {% 
            \def\PYGZus{\discretionary{\char`\_}{\Wrappedafterbreak}{\char`\_}}% 
            \def\PYGZob{\discretionary{}{\Wrappedafterbreak\char`\{}{\char`\{}}% 
            \def\PYGZcb{\discretionary{\char`\}}{\Wrappedafterbreak}{\char`\}}}% 
            \def\PYGZca{\discretionary{\char`\^}{\Wrappedafterbreak}{\char`\^}}% 
            \def\PYGZam{\discretionary{\char`\&}{\Wrappedafterbreak}{\char`\&}}% 
            \def\PYGZlt{\discretionary{}{\Wrappedafterbreak\char`\<}{\char`\<}}% 
            \def\PYGZgt{\discretionary{\char`\>}{\Wrappedafterbreak}{\char`\>}}% 
            \def\PYGZsh{\discretionary{}{\Wrappedafterbreak\char`\#}{\char`\#}}% 
            \def\PYGZpc{\discretionary{}{\Wrappedafterbreak\char`\%}{\char`\%}}% 
            \def\PYGZdl{\discretionary{}{\Wrappedafterbreak\char`\$}{\char`\$}}% 
            \def\PYGZhy{\discretionary{\char`\-}{\Wrappedafterbreak}{\char`\-}}% 
            \def\PYGZsq{\discretionary{}{\Wrappedafterbreak\textquotesingle}{\textquotesingle}}% 
            \def\PYGZdq{\discretionary{}{\Wrappedafterbreak\char`\"}{\char`\"}}% 
            \def\PYGZti{\discretionary{\char`\~}{\Wrappedafterbreak}{\char`\~}}% 
        } 
        % Some characters . , ; ? ! / are not pygmentized. 
        % This macro makes them "active" and they will insert potential linebreaks 
        \newcommand*\Wrappedbreaksatpunct {% 
            \lccode`\~`\.\lowercase{\def~}{\discretionary{\hbox{\char`\.}}{\Wrappedafterbreak}{\hbox{\char`\.}}}% 
            \lccode`\~`\,\lowercase{\def~}{\discretionary{\hbox{\char`\,}}{\Wrappedafterbreak}{\hbox{\char`\,}}}% 
            \lccode`\~`\;\lowercase{\def~}{\discretionary{\hbox{\char`\;}}{\Wrappedafterbreak}{\hbox{\char`\;}}}% 
            \lccode`\~`\:\lowercase{\def~}{\discretionary{\hbox{\char`\:}}{\Wrappedafterbreak}{\hbox{\char`\:}}}% 
            \lccode`\~`\?\lowercase{\def~}{\discretionary{\hbox{\char`\?}}{\Wrappedafterbreak}{\hbox{\char`\?}}}% 
            \lccode`\~`\!\lowercase{\def~}{\discretionary{\hbox{\char`\!}}{\Wrappedafterbreak}{\hbox{\char`\!}}}% 
            \lccode`\~`\/\lowercase{\def~}{\discretionary{\hbox{\char`\/}}{\Wrappedafterbreak}{\hbox{\char`\/}}}% 
            \catcode`\.\active
            \catcode`\,\active 
            \catcode`\;\active
            \catcode`\:\active
            \catcode`\?\active
            \catcode`\!\active
            \catcode`\/\active 
            \lccode`\~`\~ 	
        }
    \makeatother

    \let\OriginalVerbatim=\Verbatim
    \makeatletter
    \renewcommand{\Verbatim}[1][1]{%
        %\parskip\z@skip
        \sbox\Wrappedcontinuationbox {\Wrappedcontinuationsymbol}%
        \sbox\Wrappedvisiblespacebox {\FV@SetupFont\Wrappedvisiblespace}%
        \def\FancyVerbFormatLine ##1{\hsize\linewidth
            \vtop{\raggedright\hyphenpenalty\z@\exhyphenpenalty\z@
                \doublehyphendemerits\z@\finalhyphendemerits\z@
                \strut ##1\strut}%
        }%
        % If the linebreak is at a space, the latter will be displayed as visible
        % space at end of first line, and a continuation symbol starts next line.
        % Stretch/shrink are however usually zero for typewriter font.
        \def\FV@Space {%
            \nobreak\hskip\z@ plus\fontdimen3\font minus\fontdimen4\font
            \discretionary{\copy\Wrappedvisiblespacebox}{\Wrappedafterbreak}
            {\kern\fontdimen2\font}%
        }%
        
        % Allow breaks at special characters using \PYG... macros.
        \Wrappedbreaksatspecials
        % Breaks at punctuation characters . , ; ? ! and / need catcode=\active 	
        \OriginalVerbatim[#1,codes*=\Wrappedbreaksatpunct]%
    }
    \makeatother

    % Exact colors from NB
    \definecolor{incolor}{HTML}{303F9F}
    \definecolor{outcolor}{HTML}{D84315}
    \definecolor{cellborder}{HTML}{CFCFCF}
    \definecolor{cellbackground}{HTML}{F7F7F7}
    
    % prompt
    \makeatletter
    \newcommand{\boxspacing}{\kern\kvtcb@left@rule\kern\kvtcb@boxsep}
    \makeatother
    \newcommand{\prompt}[4]{
        \ttfamily\llap{{\color{#2}[#3]:\hspace{3pt}#4}}\vspace{-\baselineskip}
    }
    

    
    % Prevent overflowing lines due to hard-to-break entities
    \sloppy 
    % Setup hyperref package
    \hypersetup{
      breaklinks=true,  % so long urls are correctly broken across lines
      colorlinks=true,
      urlcolor=urlcolor,
      linkcolor=linkcolor,
      citecolor=citecolor,
      }
    % Slightly bigger margins than the latex defaults
    
    \geometry{verbose,tmargin=1in,bmargin=1in,lmargin=1in,rmargin=1in}
    
    

\begin{document}
    
\begin{center}    
    \LARGE
    \hypertarget{adv-help-mailbox-analysis}{%
\section*{ADV Help Mailbox Analysis}\label{adv-help-mailbox-analysis}}

\normalsize
Fall 2020

Author: Ethan Creagar

\end{center}

\tableofcontents
    \hypertarget{description}{%
\subsubsection{Description:}\label{description}}

This project, authorized and supervized by Nate Williams and Colorado
State University Advancement, explored the history of the ADV Help
Outlook inbox from 2016-2020 to search for insight on trends and provide
baseline numbers for the helpdesk. Specifically, the following points
will be addressed:

\begin{center}\rule{0.5\linewidth}{\linethickness}\end{center}

\begin{enumerate}
\def\labelenumi{\arabic{enumi})}
\item
  \textbf{Inbox basics:} How many tikets has the helpdesk d? On
  average, how many of these tickets do we solve? How has this changed
  over time?
\item
  \textbf{Time Trends:} What are our times of highest ticket volume?
\item
  \textbf{Category Trends:} What is the category breakdown on incoming
  tickets? What is does this breakdown look like by time period?
\item
  \textbf{Completed Ticket Trends:} What is the breakdown of tickets
  completed by category? What does this breakdown look like by
  technician?
\end{enumerate}

\begin{center}\rule{0.5\linewidth}{\linethickness}\end{center}

These questions give us insight into the workings of the helpdesk and
help us set goals and be better prepared for the issues of CSUA
employees in the future.

    \begin{center}\rule{0.5\linewidth}{\linethickness}\end{center}

\hypertarget{summary-of-findings}{%
\subsection{Summary of Findings:}\label{summary-of-findings}}

The insights we can pull from the inbox are stated below.

\hypertarget{inbox-basics}{%
\subsubsection{Inbox basics:}\label{inbox-basics}}

How many tikets has the helpdesk received? On average, how many of these
tickets do we solve? How has this changed over time?

\begin{itemize}
\tightlist
\item
  \textbf{ADV Help has received 15,377 emails over the last 5 years.}
  This is a trend that has been growing - on average we receive about 5
  more emails every month than the last. We solve about 95 percent of
  tickets (\(\pm5 \%\)).
\end{itemize}

\hypertarget{time-trends}{%
\subsubsection{Time Trends:}\label{time-trends}}

What are our times of highest ticket volume? How can we utilize staff
and resources best to fit with these times?

\begin{itemize}
\item
  \textbf{Afternoons} are typically more busy than mornings. We have
  received almost 53\% of our tickets in the afternoon, although that
  trend has changed this calendar year. In particular, we see our most
  emails between 9:00 and 11:00 AM and 1:00 and 3:00 PM 
\item
  \textbf{Tuesdays and Wednesdays} are our busiest days of the week.
  Monday afternoons are busy, while mornings have much less volume.
  Friday is our least busy time, both morning and afternoon. 
\item
  Generally, \textbf{October, August and February} are times of more
  emails, while \textbf{December} is our least-emailed month.
\end{itemize}

\hypertarget{category-trends}{%
\subsubsection{Category Trends:}\label{category-trends}}

What is the category breakdown on incoming tickets? What does this
breakdown look like by time period?

\begin{itemize}
\item
  \textbf{About 20 percent of the total tickets we receive are Hardware
  tickets, our highest category.} Email and Network tickets are next,
  with just over 8 percent each, then Software tickets. 
\item
  \textbf{As opposed to out total tickets, we receive more Remote
  tickets in the morning}. Other categories do not see this same trend. 
\item
  \textbf{Software tickets are growing at a faster rate than total
  tickets are.} We see that year by year, a higher percentage of our
  total tickets are software based. 
\item
  We tend to see a \textbf{spike in personnel tickets in August.} This
  has been a consistent trend accross the last few years. 
\item
  \textbf{We see the most email tickets in January.} This is a
  significant finding because January is generally one of our lowest
  volume times for lowest tickets. We may want to examine whether there
  is a reason this would be the case. 
\item
  \textbf{Purchasing tickets see a spike in September}. We should be on
  the lookout for more tickets involving purchasing, hardware
  replacements, etc. at this time of the year. 
\item
  \textbf{We see the most network tickets on Tuesday mornings}. It is
  not out of the ordinary to see an increase in tickets on Tuesdays, but
  we know that mornings are generally less busy than afternoons. We also
  don't see a consistancy in increased Network tickets in the mornings,
  which could lead us to believe that Tuesday morning Network issues may
  need to be examined further. This trend is most pronounced in 2020.
\end{itemize}

\hypertarget{completed-ticket-trends}{%
\subsubsection{Completed Ticket Trends:}\label{completed-ticket-trends}}

What is the breakdown of tickets completed by category? What does this
breakdown look like by technician?

\begin{itemize}
\item
  \textbf{We generally complete tickets at a similar rate to which we
  receive them in each cateogry}. The Hardware, Remote, and Software
  ticket categories have the biggest discrepencies in them between
  received and completed. 
\item
  \textbf{New hires do not have much of a difference in tickets
  completed by category than more experienced employees do.} We do see
  that (with a small sample size) new tickets complete a higher
  percentage of hardware and software tickets, whereas more experienced
  technicians complete more network tickets.
\end{itemize}


    \hypertarget{part-1---inbox-basics}{%
\section{Part 1 - Inbox Basics}\label{part-1---inbox-basics}}

In Part 1, we look to answer some basic questions about the inbox, such
as the quantity of tickets received and how this has changed over time.

\hypertarget{a-how-many-emails-have-we-received}{%
\subsubsection{a: How many emails have we
received?}\label{a-how-many-emails-have-we-received}}


    Through the last 5 years, ADV Help has received 15,377 emails.

    \hypertarget{b-on-average-how-many-tickets-do-we-solve}{%
\subsubsection{b: On average, how many tickets do we
solve?}\label{b-on-average-how-many-tickets-do-we-solve}}

    \begin{Verbatim}[commandchars=\\\{\}]
Completed 5660 out of 6008 unique tickets: 94.21\% (plus/minus 5\%)
    \end{Verbatim}

    By indexing the data to only include the tickets whose categories
contain the word ``Done'', we can see how many tickets we have
completed. We can compare this against only unique emails, defined as
emails without ``RE:'' at the beginning of them as this would indicate
that the email was part of a chain, and find the percentage of tickets
we've completed. This percentage relies on accurate ticketing of data,
so I built in a 5\% reduction and Confidence Interval of 5\% to adjust
for marking the same ticket as ``Done'' twice at different points in its
chain. \(\pm 5\%\) was chosen by exploring a subset of tickets and
examining how many were marked done at two different times in the inbox.

    \hypertarget{c-how-have-these-trends-changed-over-time}{%
\subsubsection{c: How have these trends changed over
time?}\label{c-how-have-these-trends-changed-over-time}}

    \begin{center}
    \adjustimage{max size={0.9\linewidth}{0.9\paperheight}}{output_16_0.png}
    \end{center}
    { \hspace*{\fill} \\}
    
    Here we see the trend of our emails received from February of 2016 to
April of 2020. We see that there is an upward trend in this data and we
can calculate this trend below.


    \begin{Verbatim}[commandchars=\\\{\}]
Intercept: 88.303
Beta 1: 5.226
    \end{Verbatim}

    The line of best fit through the data can be quantified as
\(\hat{y} = 88.303 + 5.226\hat{x}\). \(\beta_1\) is likely affected by
the outlier of March 2020.

\textbf{Over time, the number of emails received by ADV Help has grown
at a rate of about 5.23 emails per month, or about 63 emails per year.}

    \hypertarget{part-2-time-trends}{%
\section{Part 2: Time Trends}\label{part-2-time-trends}}

In part 2, we want to examine how many tickets we get in different time
periods, whether this be morning vs.~afternoon, months, years, or a
combination of these things.

This can help us understand inbox trends in order to allocate staff and
resources better.

\hypertarget{what-are-our-times-of-highest-volume}{%
\subsubsection{What are our times of highest
volume?}\label{what-are-our-times-of-highest-volume}}

    \textbf{Methodology}

We can plot emails received by time frame with many options, including:

\begin{itemize}
\tightlist
\item
  ``Date'': Which weekday - Monday to Sunday - the email was received 
\item
  ``AMPM'': If the email was received in the morning or afternoon 
\item
  ``Hour'': Which hour of the day - 1 to 23 - the email was received 
\item
  ``Minute'': Which minute - 1 to 60 - of the hour the email was
  received 
\item
  ``Month'': Which month - 1 to 12 - the email was received 
\item
  ``Day'': Which day - 1 to 31 - the email was received 
\item
  ``Year'': Which year the email was received in 
\end{itemize}

With this information, we can plot something simple like whether we
receive more tickets in the morning or afternoon:

    \begin{center}
    \adjustimage{max size={0.7\linewidth}{0.7\paperheight}}{output_25_0.png}
    \end{center}
    { \hspace*{\fill} \\}
    
    We see that in our history, we have received more emails after 12:00 PM
than before. This could be helpful, but a more in depth version might
include weekday in it as well to be more thorough. Including weekday,
for example, would look like this:

    \begin{center}
    \adjustimage{max size={0.9\linewidth}{0.9\paperheight}}{output_28_0.png}
    \end{center}
    { \hspace*{\fill} \\}
    
    Here we can see that amount of tickets received in the afternoon is
higher earlier in the week and trends downward as the week goes on.
Tuesdays and Wednesdays are our busiest days, while Fridays see the
least emails of any weekday. Friday mornings also tend to be busier than
Friday afternoons, the only day where this is true. Mondays, however,
show the largest difference between afternoon and morning.

    An even more in depth version might ask whether this trend has changed
over the last few years. Including year in the plot would look like
this:

    \begin{center}
    \adjustimage{max size={0.8\linewidth}{0.8\paperheight}}{output_32_0.png}
    \end{center}
    { \hspace*{\fill} \\}
    
    Although in the past we've received more emails in the afternoon, this
trend is not true for 2020. So far, we are recieving more emails in the
morning this year.

There are many interesting combinations that can give us trends of when
we should be prepared for more emails.

    \begin{center}
    \adjustimage{max size={0.9\linewidth}{0.9\paperheight}}{output_34_0.png}
    \end{center}
    { \hspace*{\fill} \\}
    
    Here, we see that March is one of our largest month for emails received.
However, this is misleading, as we see in the plot below.

    \begin{center}
    \adjustimage{max size={0.9\linewidth}{0.9\paperheight}}{output_36_0.png}
    \end{center}
    { \hspace*{\fill} \\}
    
    By organizing the variables this way, we can see trends by year
colorized by month. This allows us to see if there are any trends with
certain months being more or less busy. If we had just broken this down
by month, the data shows March as one of our busiest months, but looking
at it this way we can see that, apart from the outlier in 2020, the
opposite is usually true and March is not generally a large month for
us.

    \begin{center}
    \adjustimage{max size={0.9\linewidth}{0.9\paperheight}}{output_38_0.png}
    \end{center}
    { \hspace*{\fill} \\}
    
    This plot shows us the inverse of the last plot - each month, colorized
by which year the ticket was received in. This gives us insight to be
able to compare the first few months of a year to previous years or
specific months against previous years. We can see here that most months
seem to be rising over time, which agrees with our previous insight that
we are recieving more and more emails. We can also see that even before
lockdown, January and February were shaping up to be as busy or busier
than last year, meaning not all of the influx of emails should be
attributed to working from home and that we were likely to continue to
see growth in the inbox regardless.

    \begin{center}
    \adjustimage{max size={0.9\linewidth}{0.9\paperheight}}{output_40_0.png}
    \end{center}
    { \hspace*{\fill} \\}
    
    Here we break the last two down further by splitting the groups into
Year, morning and afternoon. We can see use this plot to see whether
changes over time are consistant between morning and afternoon. We can
see that in some months, like October, we have been historically more
likely to receive emails in the afternoon.

    More examples of plots with insights are shown below.

    \begin{center}
    \adjustimage{max size={0.9\linewidth}{0.9\paperheight}}{output_43_0.png}
    \end{center}
    { \hspace*{\fill} \\}
    
    \begin{itemize}
\tightlist
\item
  \textbf{Total Emails - Hour:} We receive the most emails between 9 and
  10 o'clock, followed by between 10 and 11 o'clock and between 2 and 3
  o'clock.
\item
  \textbf{Total Emails - Hour and Weekday:} The amount of emails we
  receive by hour is fairly constant across the day of the week. We see
  more emails between 9:00 and 12:00 on Tuesdays than any other day,
  while we see the most emails between 2:00 and 4:00 on Wednesdays.
\item
  \textbf{Total Emails - Morning vs.~Afternoon by Day of the Month:} We
  see a similar breakdown to the plot below here, but we can see that we
  get more emails in the afternoon later in the month than we do earlier
  in the month.
\item
  \textbf{Total Emails - Day of the Month:} We see the most emails on
  the 5th and 6th of the month, followed by the 30th. We receive
  slightly less emails later in the month than earlier in the month, but
  it's not a very significant difference.
\item
  \textbf{Total Emails - Year:} We have looked at this in different
  ways, but this plot shows us that this year will be our highest
  emailed year by far.
\item
  \textbf{Total Emails = Morning vs.~Afternooon by Month:} we see more
  email in the afternoon in the later months than we do in the first 6
  months.
\end{itemize}

    \begin{center}\rule{0.5\linewidth}{\linethickness}\end{center}

There are many more combinations of plots we could create, but the
insight drawn from these is valuable.

To answer our question, we can summarize some \textbf{trends of the
times of increased and decreased volume:}

\begin{itemize}
\item
  \textbf{Afternoons} are typically more busy than mornings. We have
  received almost 53\% of our tickets in the afternoon, although that
  trend has changed this calendar year. In particular, we see our most
  emails between 9:00 and 11:00 AM and 1:00 and 3:00 PM 
\item
  \textbf{Tuesdays and Wednesdays} are our busiest days of the week.
  Monday afternoons are busy, while mornings have much less volume.
  Friday is our least busy time, both morning and afternoon. 
\item
  \textbf{Earlier in the Month} tends to be slightly busier than later
  in the Month. 
\item
  \textbf{The Month of the year} in which we received the most email
  does not have much of a constant trend. Generally, \textbf{October,
  August and February} are times of more emails, while \textbf{December}
  is our least-emailed month.
\end{itemize}

    \hypertarget{part-3-category-trends}{%
\section{Part 3: Category Trends}\label{part-3-category-trends}}

In this section, we break our inbox down into categories to see if we
can find any trends based on the different categories. We look to
explore this by different time period like we did in part 2.

\hypertarget{a-what-is-the-category-breakdown-on-incoming-tickets}{%
\subsection{a: What is the category breakdown on incoming
tickets?}\label{a-what-is-the-category-breakdown-on-incoming-tickets}}

    Using a function to sort emails, we can go through each subject line and
find keywords that might mean a ticket is likely to be in a certain
category. Some examples of these keywords are as follows:

    \textbf{Email} Tickets: ``outlook'', ``email'', ``mailbox'', `delegate',
`calendar', `spam'

\textbf{Remote} Tickets: ``remote'', ``rdp'', ``remote desktop'',
``pulse'', ``duo'', ``vpn''

\textbf{Hardware}: ``monitor'', ``webcam'', ``keyboard'', ``dock'',
``mouse'', ``cables'', ``printer''

\textbf{Network}: ``Network drives'', ``internet'', ``connection'',
``domain'', ``wifi'', ``distribution list''

\textbf{Adobe}: ``adobe'', ``acrobat'', ``pdf'', ``creative cloud'',
``photoshop''

\textbf{Software}: ``install'', ``download'', ``activation key'',
``software'', ``licensing'', ``word'', ``excel''

\textbf{Purchasing}: ``order'', ``delivery'', ``kuali'', ``cdw-g'',
``pricing''

\textbf{Personnel}: ``employee'', ``hire'', ``on-boarding'',
``resignation'', ``departures''

\textbf{Phone}: ``call'', ``missed call'', ``voicemail'', ``phone''

We can plot Categories by time or against one
another by time. For example, below is a plot of Email category tickets
vs Hardware category tickets over the last 4 years.

    \begin{center}
    \adjustimage{max size={0.9\linewidth}{0.9\paperheight}}{output_55_0.png}
    \end{center}
    { \hspace*{\fill} \\}
    
    \begin{center}
    \adjustimage{max size={0.9\linewidth}{0.9\paperheight}}{output_56_0.png}
    \end{center}
    { \hspace*{\fill} \\}
    
    The rest of the categories are plotted above. We can see that most of
them have a spike recently and are generally trending upward. Most of
these spikes are easily explainable - for instance, the \textbf{Remote}
category spikes around March of 2020, when the division was starting to
work from home and needed our remote connection guides and software.
\textbf{Adobe} has a spike in November/December of 2019, the month when
the technicians were upgrading all machines to the new Adobe Acrobat
2017.

    The breakdown of total emails by subject can be seen below.

            \begin{tcolorbox}[breakable, size=fbox, boxrule=.5pt, pad at break*=1mm, opacityfill=0]
\prompt{Out}{outcolor}{36}{\boxspacing}
\begin{Verbatim}[commandchars=\\\{\}]
            From  Total  Percentage
Other       5584  15251        36.6
Adobe        464  15251         3.0
Email       1264  15251         8.3
Hardware    3245  15251        21.3
Network     1324  15251         8.7
Personnel    497  15251         3.3
Phone       1124  15251         7.4
Purchasing   124  15251         0.8
Remote       624  15251         4.1
Software    1001  15251         6.6
\end{Verbatim}
\end{tcolorbox}
        
    \hypertarget{b-what-does-this-breakdown-look-like-by-time-period}{%
\subsection{b: What does this breakdown look like by time
period?}\label{b-what-does-this-breakdown-look-like-by-time-period}}

We might want to know if there are any trends of categorized tickets to
know if we see more errors or slowness at any times in specific.

    Our Group Categories and Plot Categories functions can give us the same
insight as we received above, but this time broken down into categories
so we can see when we are recieving the most of a certain type of
category in a more detailed way than just over each month like above.


    \begin{center}
    \adjustimage{max size={0.9\linewidth}{0.9\paperheight}}{output_65_0.png}
    \end{center}
    { \hspace*{\fill} \\}
    
    For example, here we see the \textbf{Hardware} category broken down by
month. Again, this may not be as insightful as we'd like and we might
have a bit of a bias toward March and April due to the large spike of
emails in that time period of 2020. We can break this down further to
get a better idea of this category's time distribution.

    \begin{center}
    \adjustimage{max size={0.9\linewidth}{0.9\paperheight}}{output_67_0.png}
    \end{center}
    { \hspace*{\fill} \\}
    
    We see here that we are correct - There was a big spike in February and
March of 2020. We could, however, argue whether the February spike was
out of the ordinary based on the fact that \textbf{we seem to see many
Hardware tickets in February each year. We also seem to see spikes in
October/November}.

    \begin{center}
    \adjustimage{max size={0.7\linewidth}{0.7\paperheight}}{output_69_0.png}
    \end{center}
    { \hspace*{\fill} \\}
    
    The first plot breaks down each year by month so we can see trends by
month inside each year, while the second breaks down each month by year
to see how our number of emails received in each month have grown or
shrunk by year. This gives us the insight that \textbf{like emails in
general and most categories of emails, Hardware tickets are rising farly
consistently by year}.

    \begin{center}
    \adjustimage{max size={0.9\linewidth}{0.9\paperheight}}{output_71_0.png}
    \end{center}
    { \hspace*{\fill} \\}
    
    This plot tells us that \textbf{certain months, such as October and
November, and have generally had a larger discrepancy between the
morning and evening emails.}

    Now that we understand the basics of examining categorized tickets by
time frame, we can plot each category against our overall inbox. By
doing this, we can see any differences that stand out in any of the
categories. We can break these time frames down by time periods that
might be of interest, like Year, Month, Weekday, and AM/PM.

    \begin{center}
    \adjustimage{max size={0.8\linewidth}{0.8\paperheight}}{output_74_0.png}

    \adjustimage{max size={0.8\linewidth}{0.8\paperheight}}{output_74_1.png}
    \end{center}
    { \hspace*{\fill} \\}
    
    In this plot, we can see that most categories seem to follow the same
trend as our overall inbox of receiving more emails in the afternoon.
However, Remote tickets (tickets containing keywords such as ``Pulse'',
``VPN'', ``Duo'', and ``Remote Desktop'' are received more often in the
mornings. This can give us the insight that \textbf{we should expect
more tickets about remote connection in the mornings.}

    We can create similar plots for Year and Month Trends:

\newpage

    \begin{center}
    \adjustimage{max size={0.7\linewidth}{0.7\paperheight}}{output_77_0.png}

    \adjustimage{max size={0.9\linewidth}{0.9\paperheight}}{output_77_1.png}
    \end{center}
    { \hspace*{\fill} \\}
    
    These plots contain much more information. We can see that our general
trends see August - October/November as the high points of the year.
this is also true for many of our categories, but some differences that
stand out between the plots include: While most other plots show 2020 as
our high point, \textbf{Phone tickets do not seem to be rising at the
same rate as other tickets}. \textbf{Software (including keywords
``download'', ``install'', ``license'', etc.) tickets have seen the
biggest rise of any category this year}.

    
    We've seen that we get a lot more Remote tickets in the mornings than we do in the
afternoons, which could help us be more on the lookout for these tickets
in the mornings. We can see that this trend is also heavily influenced
by 2020.

    \begin{center}
    \adjustimage{max size={0.9\linewidth}{0.9\paperheight}}{output_83_0.png}
    \end{center}
    { \hspace*{\fill} \\}

    \begin{center}
    \adjustimage{max size={0.9\linewidth}{0.9\paperheight}}{output_84_0.png}
    \end{center}
    { \hspace*{\fill} \\}
    
Examining many more plots and searching for interesting and/or unexpected
trends yeilded the insights explored in the plots below.

    \begin{center}
    \adjustimage{max size={0.9\linewidth}{0.9\paperheight}}{output_86_0.png}
    \end{center}
    { \hspace*{\fill} \\}
    
    \begin{itemize}
\item
  \textbf{Software Tickets}: Plotting software tickets vs total tickets
  by year received, we can see that \textbf{software tickets}, tickets
  including keywords such as ``download'', ``update'', ``software'',
  ``installation'', and ``licensing'', \textbf{seem to be growing at a
  faster rate than our tickets overall}. As a total percentage of our
  tickets, software tickets are also growing year-by-year with a spike
  in 2020.
\item
  \textbf{Personnel tickets}: \textbf{We see that we get a spike in
  personnel tickets,} tickets including keywords such as ``Hire'',
  ``On-boarding'', ``Welcome'', and ``Resignation'', \textbf{in August}.
  Looking at the plot of Personnel tickets by year broken into month
  received, we can see that \textbf{this spike in August/September is
  farily consistant over at least the last three years,} so this spike
  is fairly reliable.
\end{itemize}

    \begin{center}
    \adjustimage{max size={0.9\linewidth}{0.9\paperheight}}{output_88_0.png}
    \end{center}
    { \hspace*{\fill} \\}
    
    \begin{itemize}
\item
  \textbf{We see the most network tickets on Tuesday mornings}. It is
  not out of the ordinary to see an increase in tickets on Tuesdays, but
  we know that mornings are generally less busy than afternoons.
\item
  \textbf{Email tickets by month received:} By breaking down our
  ``email'' tickets, tickets with keywords like ``Outlook'',
  ``Mailbox'', and ``Email'', we can see that we see our largest spike
  in January. This may be an interesting trend to look into as it
  counters our findings above that January is generally not a time of
  large email volume.
\item
  \textbf{Purchasing tickets by month received}: Our purchasing tickets,
  tickets with keywords such as ``order'', ``purchase'',
  ``replacement'', or ``delivery'', sees by far its largest spike in
  September.
\end{itemize}

    \begin{center}\rule{0.5\linewidth}{\linethickness}\end{center}

    \hypertarget{part-4-completed-ticket-trends}{%
\section{Part 4: Completed Ticket
Trends}\label{part-4-completed-ticket-trends}}

In part 4, we want to answer questions about the breakdown of our total
tickets in each category. We can use this information to inform
descisions about what tickets to devote the most resources to, how we
should train employees, and more.

\hypertarget{a-what-is-the-breakdown-of-tickets-completed-by-category}{%
\subsection{a: What is the breakdown of tickets completed by
category?}\label{a-what-is-the-breakdown-of-tickets-completed-by-category}}

    We now have a breakdown of what tickets we receive the most of and when,
which can be useful to know what kind of tickets we should devote the
most resources to. It might also be interesting to see what kind of
tickets we complete at the highest rate to see if we struggle in some
categories more than others. Note that these percentages lean on the
side of too high due to tickets marked Done twice.


    \begin{Verbatim}[commandchars=\\\{\}]
            Percent - Completed  Percent - Total
Adobe                      2.97              3.0
Email                      8.37              8.3
Hardware                  19.26             21.3
Network                    8.73              8.7
Personnel                  3.69              3.3
Phone                      7.56              7.4
Purchasing                 0.81              0.8
Remote                     3.15              4.1
Software                   5.31              6.6

    \end{Verbatim}

    We add some noise to the Completed tickets group to account for the
lower percentage of Uncategorized tickets and plot the two data frames
side-by-side.

    \begin{center}
    \adjustimage{max size={0.9\linewidth}{0.9\paperheight}}{output_98_0.png}
    \end{center}
    { \hspace*{\fill} \\}
    
    We can see that our numbers follow pretty closely, and with the
inaccuracy of the done tickets we can say that most of the percentages
are likely very close to each other without duplicates. The interesting
columns then are the ones that have lower or very close percentages in
the Done data frame. These are \textbf{Hardware, Remote, and Software.}
If our data is accurate, this would mean that \textbf{these are the
tickets that go uncompleted most often}. This could be something to look
out for in the future.

    \hypertarget{b-what-is-the-breakdown-of-completed-tickets-by-technician}{%
\subsection{b: What is the breakdown of completed tickets by
technician?}\label{b-what-is-the-breakdown-of-completed-tickets-by-technician}}


    Dividing the tickets by technician, we can compare what kind of tickets
a technician responds to most vs what kind of tickets we receive the
most to see if our technicians have any obvious gaps in the type of
tickets they respond to. For instance, here are the number of tickets
I've completed (those marked as Done - Ethan) vs the total numbers.


    \begin{Verbatim}[commandchars=\\\{\}]
            Percents - Ethan  Percents - Total
Adobe                    7.8               3.0
Email                    9.5               8.3
Hardware                23.1              21.3
Network                  7.4               8.7
Personnel                3.7               3.3
Phone                    5.6               7.4
Purchasing               1.7               0.8
Remote                   6.9               4.1
Software                 5.7               6.6
    \end{Verbatim}

    We can see that my responses (on top) have a bit of an advantage in the
Adobe and Email categories, which means I respond to tickets of those
sort at a higher rate than our historical average. However, I respond at
a lower percentage to Network, Phone, and Software category tickets,
meaning I may have a bit of work to do in terms of finishing those
tickets.

    We can gather this data for any technician whose tickets are still
marked ``Done - Technician'' in our inbox. For example, here we can see
Nate's tickets:

    \begin{Verbatim}[commandchars=\\\{\}]
            Percent - Nate  Percent - Total
Adobe                  2.4              3.0
Email                  9.0              8.3
Hardware              19.5             21.3
Network               10.7              8.7
Personnel              5.4              3.3
Phone                 10.2              7.4
Purchasing             0.5              0.8
Remote                 2.8              4.1
Software               5.6              6.6

    \end{Verbatim}
 
    Comparing the emails from Done - Nate to the total ticket breakdown, we
can see that his responses follow the distribution pretty closely. This
would make sense, since he's solved about 1/6 of the total ADV Help
tickets. Nate's reponses also lean toward Network tickets slightly,
which we would expect with what kind of tickets he tends to take care
of.


    \textbf{Training} is one of the areas that help-desks might look to
improve upon. We can use the style of data frame above to gather
information about how effectively technicians have been trained, and
which areas of training may still need to be improved upon. Below are
three Technicians. Agent 1 worked in the Helpdesk from February of 2016
to May of 2019. Agent 2 has worked in the Helpdesk from October of 2018
to present, and Agent 3 has worked in the Helpdesk from October 2019 to
present.

    \begin{Verbatim}[commandchars=\\\{\}]
                  Percent - Agent 1  Percent - Agent2  Percent - Agent3
Email Categories
Adobe                           2.1               7.8               4.5
Email                           9.9               9.5               9.9
Hardware                       22.4              23.1              25.8
Network                        10.6               7.4               8.2
Personnel                       3.7               3.7               2.6
Phone                           9.2               5.6               5.1
Purchasing                      1.3               1.7               0.7
Remote                          2.6               6.9               5.1
Software                        3.8               5.7              11.8
    \end{Verbatim}

    \begin{center}
    \adjustimage{max size={0.9\linewidth}{0.9\paperheight}}{output_111_0.png}
    \end{center}
    { \hspace*{\fill} \\}
    
    Due to these time differences, this can give us insights about training.
For instance, we see that the percentage of Network tickets, tickets
with keyworks such as ``internet'', ``wifi'', ``domain'',
``permissions'', ``drives'', and ``network drives'', rises from Agent 3
and 2 to Agent 1. this shows us that with more time, agents may get more
comfortable with network tickets or be asked to answer more of them. We
see a similar trend with Phone tickets. On the opposite, we see a
category such as hardware tickets, tickets including ``station'',
``setup'', and other computer and printer hardware words. We can see
that this category gets smaller as the agents answer more tickets, this
could show us that as agents are further trained, they are called upon
less to answer tickets dealing with hardware.

Overall, the categories are fairly even, showing us that although
helpdesks may get more comfortable with certain types of tickets over
time, \textbf{training is doing a good job of preparing new hires to
deal with all types of tickets.}

\begin{center}\rule{0.5\linewidth}{\linethickness}\end{center}

To conclude our question about completed tickets broken down by
category, we can say that we tend to complete tickets at about the same
rate as we receive them in each category. Our biggest discrepencies are
in the Hardware, Software and Remote categories. We can also make
conclusions about our quality of training based on this breakdown. In
particular, we can say, with a small sample size, that newer hires tend
to complete tickets at about the same rate across categories that more
experienced technicians do. We do see an increased amount of Hardware
tickets and software tickets completed by newer hires than older hires,
and more network tickets completed by older hires.



    % Add a bibliography block to the postdoc
    
\end{document}
