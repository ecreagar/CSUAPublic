\documentclass[11pt]{article}

    \usepackage[breakable]{tcolorbox}
    \usepackage{parskip} % Stop auto-indenting (to mimic markdown behaviour)
    
    \usepackage{iftex}
    \ifPDFTeX
    	\usepackage[T1]{fontenc}
    	\usepackage{mathpazo}
    \else
    	\usepackage{fontspec}
    \fi

    % Basic figure setup, for now with no caption control since it's done
    % automatically by Pandoc (which extracts ![](path) syntax from Markdown).
    \usepackage{graphicx}
    % Maintain compatibility with old templates. Remove in nbconvert 6.0
    \let\Oldincludegraphics\includegraphics
    % Ensure that by default, figures have no caption (until we provide a
    % proper Figure object with a Caption API and a way to capture that
    % in the conversion process - todo).
    \usepackage{caption}
    \DeclareCaptionFormat{nocaption}{}
    \captionsetup{format=nocaption,aboveskip=0pt,belowskip=0pt}

    \usepackage[Export]{adjustbox} % Used to constrain images to a maximum size
    \adjustboxset{max size={0.9\linewidth}{0.9\paperheight}}
    \usepackage{float}
    \floatplacement{figure}{H} % forces figures to be placed at the correct location
    \usepackage{xcolor} % Allow colors to be defined
    \usepackage{enumerate} % Needed for markdown enumerations to work
    \usepackage{geometry} % Used to adjust the document margins
    \usepackage{amsmath} % Equations
    \usepackage{amssymb} % Equations
    \usepackage{textcomp} % defines textquotesingle
    % Hack from http://tex.stackexchange.com/a/47451/13684:
    \AtBeginDocument{%
        \def\PYZsq{\textquotesingle}% Upright quotes in Pygmentized code
    }
    \usepackage{upquote} % Upright quotes for verbatim code
    \usepackage{eurosym} % defines \euro
    \usepackage[mathletters]{ucs} % Extended unicode (utf-8) support
    \usepackage{fancyvrb} % verbatim replacement that allows latex
    \usepackage{grffile} % extends the file name processing of package graphics 
                         % to support a larger range
    \makeatletter % fix for grffile with XeLaTeX
    \def\Gread@@xetex#1{%
      \IfFileExists{"\Gin@base".bb}%
      {\Gread@eps{\Gin@base.bb}}%
      {\Gread@@xetex@aux#1}%
    }
    \makeatother

    % The hyperref package gives us a pdf with properly built
    % internal navigation ('pdf bookmarks' for the table of contents,
    % internal cross-reference links, web links for URLs, etc.)
    \usepackage{hyperref}
    % The default LaTeX title has an obnoxious amount of whitespace. By default,
    % titling removes some of it. It also provides customization options.
    \usepackage{titling}
    \usepackage{longtable} % longtable support required by pandoc >1.10
    \usepackage{booktabs}  % table support for pandoc > 1.12.2
    \usepackage[inline]{enumitem} % IRkernel/repr support (it uses the enumerate* environment)
    \usepackage[normalem]{ulem} % ulem is needed to support strikethroughs (\sout)
                                % normalem makes italics be italics, not underlines
    \usepackage{mathrsfs}
    

    
    % Colors for the hyperref package
    \definecolor{urlcolor}{rgb}{0,.145,.698}
    \definecolor{linkcolor}{rgb}{.71,0.21,0.01}
    \definecolor{citecolor}{rgb}{.12,.54,.11}

    % ANSI colors
    \definecolor{ansi-black}{HTML}{3E424D}
    \definecolor{ansi-black-intense}{HTML}{282C36}
    \definecolor{ansi-red}{HTML}{E75C58}
    \definecolor{ansi-red-intense}{HTML}{B22B31}
    \definecolor{ansi-green}{HTML}{00A250}
    \definecolor{ansi-green-intense}{HTML}{007427}
    \definecolor{ansi-yellow}{HTML}{DDB62B}
    \definecolor{ansi-yellow-intense}{HTML}{B27D12}
    \definecolor{ansi-blue}{HTML}{208FFB}
    \definecolor{ansi-blue-intense}{HTML}{0065CA}
    \definecolor{ansi-magenta}{HTML}{D160C4}
    \definecolor{ansi-magenta-intense}{HTML}{A03196}
    \definecolor{ansi-cyan}{HTML}{60C6C8}
    \definecolor{ansi-cyan-intense}{HTML}{258F8F}
    \definecolor{ansi-white}{HTML}{C5C1B4}
    \definecolor{ansi-white-intense}{HTML}{A1A6B2}
    \definecolor{ansi-default-inverse-fg}{HTML}{FFFFFF}
    \definecolor{ansi-default-inverse-bg}{HTML}{000000}

    % commands and environments needed by pandoc snippets
    % extracted from the output of `pandoc -s`
    \providecommand{\tightlist}{%
      \setlength{\itemsep}{0pt}\setlength{\parskip}{0pt}}
    \DefineVerbatimEnvironment{Highlighting}{Verbatim}{commandchars=\\\{\}}
    % Add ',fontsize=\small' for more characters per line
    \newenvironment{Shaded}{}{}
    \newcommand{\KeywordTok}[1]{\textcolor[rgb]{0.00,0.44,0.13}{\textbf{{#1}}}}
    \newcommand{\DataTypeTok}[1]{\textcolor[rgb]{0.56,0.13,0.00}{{#1}}}
    \newcommand{\DecValTok}[1]{\textcolor[rgb]{0.25,0.63,0.44}{{#1}}}
    \newcommand{\BaseNTok}[1]{\textcolor[rgb]{0.25,0.63,0.44}{{#1}}}
    \newcommand{\FloatTok}[1]{\textcolor[rgb]{0.25,0.63,0.44}{{#1}}}
    \newcommand{\CharTok}[1]{\textcolor[rgb]{0.25,0.44,0.63}{{#1}}}
    \newcommand{\StringTok}[1]{\textcolor[rgb]{0.25,0.44,0.63}{{#1}}}
    \newcommand{\CommentTok}[1]{\textcolor[rgb]{0.38,0.63,0.69}{\textit{{#1}}}}
    \newcommand{\OtherTok}[1]{\textcolor[rgb]{0.00,0.44,0.13}{{#1}}}
    \newcommand{\AlertTok}[1]{\textcolor[rgb]{1.00,0.00,0.00}{\textbf{{#1}}}}
    \newcommand{\FunctionTok}[1]{\textcolor[rgb]{0.02,0.16,0.49}{{#1}}}
    \newcommand{\RegionMarkerTok}[1]{{#1}}
    \newcommand{\ErrorTok}[1]{\textcolor[rgb]{1.00,0.00,0.00}{\textbf{{#1}}}}
    \newcommand{\NormalTok}[1]{{#1}}
    
    % Additional commands for more recent versions of Pandoc
    \newcommand{\ConstantTok}[1]{\textcolor[rgb]{0.53,0.00,0.00}{{#1}}}
    \newcommand{\SpecialCharTok}[1]{\textcolor[rgb]{0.25,0.44,0.63}{{#1}}}
    \newcommand{\VerbatimStringTok}[1]{\textcolor[rgb]{0.25,0.44,0.63}{{#1}}}
    \newcommand{\SpecialStringTok}[1]{\textcolor[rgb]{0.73,0.40,0.53}{{#1}}}
    \newcommand{\ImportTok}[1]{{#1}}
    \newcommand{\DocumentationTok}[1]{\textcolor[rgb]{0.73,0.13,0.13}{\textit{{#1}}}}
    \newcommand{\AnnotationTok}[1]{\textcolor[rgb]{0.38,0.63,0.69}{\textbf{\textit{{#1}}}}}
    \newcommand{\CommentVarTok}[1]{\textcolor[rgb]{0.38,0.63,0.69}{\textbf{\textit{{#1}}}}}
    \newcommand{\VariableTok}[1]{\textcolor[rgb]{0.10,0.09,0.49}{{#1}}}
    \newcommand{\ControlFlowTok}[1]{\textcolor[rgb]{0.00,0.44,0.13}{\textbf{{#1}}}}
    \newcommand{\OperatorTok}[1]{\textcolor[rgb]{0.40,0.40,0.40}{{#1}}}
    \newcommand{\BuiltInTok}[1]{{#1}}
    \newcommand{\ExtensionTok}[1]{{#1}}
    \newcommand{\PreprocessorTok}[1]{\textcolor[rgb]{0.74,0.48,0.00}{{#1}}}
    \newcommand{\AttributeTok}[1]{\textcolor[rgb]{0.49,0.56,0.16}{{#1}}}
    \newcommand{\InformationTok}[1]{\textcolor[rgb]{0.38,0.63,0.69}{\textbf{\textit{{#1}}}}}
    \newcommand{\WarningTok}[1]{\textcolor[rgb]{0.38,0.63,0.69}{\textbf{\textit{{#1}}}}}
    
    
    % Define a nice break command that doesn't care if a line doesn't already
    % exist.
    \def\br{\hspace*{\fill} \\* }
    % Math Jax compatibility definitions
    \def\gt{>}
    \def\lt{<}
    \let\Oldtex\TeX
    \let\Oldlatex\LaTeX
    \renewcommand{\TeX}{\textrm{\Oldtex}}
    \renewcommand{\LaTeX}{\textrm{\Oldlatex}}
    % Document parameters
    % Document title
    \title{ADV Help Mailbox Analysis}
    
    
    
    
    
% Pygments definitions
\makeatletter
\def\PY@reset{\let\PY@it=\relax \let\PY@bf=\relax%
    \let\PY@ul=\relax \let\PY@tc=\relax%
    \let\PY@bc=\relax \let\PY@ff=\relax}
\def\PY@tok#1{\csname PY@tok@#1\endcsname}
\def\PY@toks#1+{\ifx\relax#1\empty\else%
    \PY@tok{#1}\expandafter\PY@toks\fi}
\def\PY@do#1{\PY@bc{\PY@tc{\PY@ul{%
    \PY@it{\PY@bf{\PY@ff{#1}}}}}}}
\def\PY#1#2{\PY@reset\PY@toks#1+\relax+\PY@do{#2}}

\expandafter\def\csname PY@tok@w\endcsname{\def\PY@tc##1{\textcolor[rgb]{0.73,0.73,0.73}{##1}}}
\expandafter\def\csname PY@tok@c\endcsname{\let\PY@it=\textit\def\PY@tc##1{\textcolor[rgb]{0.25,0.50,0.50}{##1}}}
\expandafter\def\csname PY@tok@cp\endcsname{\def\PY@tc##1{\textcolor[rgb]{0.74,0.48,0.00}{##1}}}
\expandafter\def\csname PY@tok@k\endcsname{\let\PY@bf=\textbf\def\PY@tc##1{\textcolor[rgb]{0.00,0.50,0.00}{##1}}}
\expandafter\def\csname PY@tok@kp\endcsname{\def\PY@tc##1{\textcolor[rgb]{0.00,0.50,0.00}{##1}}}
\expandafter\def\csname PY@tok@kt\endcsname{\def\PY@tc##1{\textcolor[rgb]{0.69,0.00,0.25}{##1}}}
\expandafter\def\csname PY@tok@o\endcsname{\def\PY@tc##1{\textcolor[rgb]{0.40,0.40,0.40}{##1}}}
\expandafter\def\csname PY@tok@ow\endcsname{\let\PY@bf=\textbf\def\PY@tc##1{\textcolor[rgb]{0.67,0.13,1.00}{##1}}}
\expandafter\def\csname PY@tok@nb\endcsname{\def\PY@tc##1{\textcolor[rgb]{0.00,0.50,0.00}{##1}}}
\expandafter\def\csname PY@tok@nf\endcsname{\def\PY@tc##1{\textcolor[rgb]{0.00,0.00,1.00}{##1}}}
\expandafter\def\csname PY@tok@nc\endcsname{\let\PY@bf=\textbf\def\PY@tc##1{\textcolor[rgb]{0.00,0.00,1.00}{##1}}}
\expandafter\def\csname PY@tok@nn\endcsname{\let\PY@bf=\textbf\def\PY@tc##1{\textcolor[rgb]{0.00,0.00,1.00}{##1}}}
\expandafter\def\csname PY@tok@ne\endcsname{\let\PY@bf=\textbf\def\PY@tc##1{\textcolor[rgb]{0.82,0.25,0.23}{##1}}}
\expandafter\def\csname PY@tok@nv\endcsname{\def\PY@tc##1{\textcolor[rgb]{0.10,0.09,0.49}{##1}}}
\expandafter\def\csname PY@tok@no\endcsname{\def\PY@tc##1{\textcolor[rgb]{0.53,0.00,0.00}{##1}}}
\expandafter\def\csname PY@tok@nl\endcsname{\def\PY@tc##1{\textcolor[rgb]{0.63,0.63,0.00}{##1}}}
\expandafter\def\csname PY@tok@ni\endcsname{\let\PY@bf=\textbf\def\PY@tc##1{\textcolor[rgb]{0.60,0.60,0.60}{##1}}}
\expandafter\def\csname PY@tok@na\endcsname{\def\PY@tc##1{\textcolor[rgb]{0.49,0.56,0.16}{##1}}}
\expandafter\def\csname PY@tok@nt\endcsname{\let\PY@bf=\textbf\def\PY@tc##1{\textcolor[rgb]{0.00,0.50,0.00}{##1}}}
\expandafter\def\csname PY@tok@nd\endcsname{\def\PY@tc##1{\textcolor[rgb]{0.67,0.13,1.00}{##1}}}
\expandafter\def\csname PY@tok@s\endcsname{\def\PY@tc##1{\textcolor[rgb]{0.73,0.13,0.13}{##1}}}
\expandafter\def\csname PY@tok@sd\endcsname{\let\PY@it=\textit\def\PY@tc##1{\textcolor[rgb]{0.73,0.13,0.13}{##1}}}
\expandafter\def\csname PY@tok@si\endcsname{\let\PY@bf=\textbf\def\PY@tc##1{\textcolor[rgb]{0.73,0.40,0.53}{##1}}}
\expandafter\def\csname PY@tok@se\endcsname{\let\PY@bf=\textbf\def\PY@tc##1{\textcolor[rgb]{0.73,0.40,0.13}{##1}}}
\expandafter\def\csname PY@tok@sr\endcsname{\def\PY@tc##1{\textcolor[rgb]{0.73,0.40,0.53}{##1}}}
\expandafter\def\csname PY@tok@ss\endcsname{\def\PY@tc##1{\textcolor[rgb]{0.10,0.09,0.49}{##1}}}
\expandafter\def\csname PY@tok@sx\endcsname{\def\PY@tc##1{\textcolor[rgb]{0.00,0.50,0.00}{##1}}}
\expandafter\def\csname PY@tok@m\endcsname{\def\PY@tc##1{\textcolor[rgb]{0.40,0.40,0.40}{##1}}}
\expandafter\def\csname PY@tok@gh\endcsname{\let\PY@bf=\textbf\def\PY@tc##1{\textcolor[rgb]{0.00,0.00,0.50}{##1}}}
\expandafter\def\csname PY@tok@gu\endcsname{\let\PY@bf=\textbf\def\PY@tc##1{\textcolor[rgb]{0.50,0.00,0.50}{##1}}}
\expandafter\def\csname PY@tok@gd\endcsname{\def\PY@tc##1{\textcolor[rgb]{0.63,0.00,0.00}{##1}}}
\expandafter\def\csname PY@tok@gi\endcsname{\def\PY@tc##1{\textcolor[rgb]{0.00,0.63,0.00}{##1}}}
\expandafter\def\csname PY@tok@gr\endcsname{\def\PY@tc##1{\textcolor[rgb]{1.00,0.00,0.00}{##1}}}
\expandafter\def\csname PY@tok@ge\endcsname{\let\PY@it=\textit}
\expandafter\def\csname PY@tok@gs\endcsname{\let\PY@bf=\textbf}
\expandafter\def\csname PY@tok@gp\endcsname{\let\PY@bf=\textbf\def\PY@tc##1{\textcolor[rgb]{0.00,0.00,0.50}{##1}}}
\expandafter\def\csname PY@tok@go\endcsname{\def\PY@tc##1{\textcolor[rgb]{0.53,0.53,0.53}{##1}}}
\expandafter\def\csname PY@tok@gt\endcsname{\def\PY@tc##1{\textcolor[rgb]{0.00,0.27,0.87}{##1}}}
\expandafter\def\csname PY@tok@err\endcsname{\def\PY@bc##1{\setlength{\fboxsep}{0pt}\fcolorbox[rgb]{1.00,0.00,0.00}{1,1,1}{\strut ##1}}}
\expandafter\def\csname PY@tok@kc\endcsname{\let\PY@bf=\textbf\def\PY@tc##1{\textcolor[rgb]{0.00,0.50,0.00}{##1}}}
\expandafter\def\csname PY@tok@kd\endcsname{\let\PY@bf=\textbf\def\PY@tc##1{\textcolor[rgb]{0.00,0.50,0.00}{##1}}}
\expandafter\def\csname PY@tok@kn\endcsname{\let\PY@bf=\textbf\def\PY@tc##1{\textcolor[rgb]{0.00,0.50,0.00}{##1}}}
\expandafter\def\csname PY@tok@kr\endcsname{\let\PY@bf=\textbf\def\PY@tc##1{\textcolor[rgb]{0.00,0.50,0.00}{##1}}}
\expandafter\def\csname PY@tok@bp\endcsname{\def\PY@tc##1{\textcolor[rgb]{0.00,0.50,0.00}{##1}}}
\expandafter\def\csname PY@tok@fm\endcsname{\def\PY@tc##1{\textcolor[rgb]{0.00,0.00,1.00}{##1}}}
\expandafter\def\csname PY@tok@vc\endcsname{\def\PY@tc##1{\textcolor[rgb]{0.10,0.09,0.49}{##1}}}
\expandafter\def\csname PY@tok@vg\endcsname{\def\PY@tc##1{\textcolor[rgb]{0.10,0.09,0.49}{##1}}}
\expandafter\def\csname PY@tok@vi\endcsname{\def\PY@tc##1{\textcolor[rgb]{0.10,0.09,0.49}{##1}}}
\expandafter\def\csname PY@tok@vm\endcsname{\def\PY@tc##1{\textcolor[rgb]{0.10,0.09,0.49}{##1}}}
\expandafter\def\csname PY@tok@sa\endcsname{\def\PY@tc##1{\textcolor[rgb]{0.73,0.13,0.13}{##1}}}
\expandafter\def\csname PY@tok@sb\endcsname{\def\PY@tc##1{\textcolor[rgb]{0.73,0.13,0.13}{##1}}}
\expandafter\def\csname PY@tok@sc\endcsname{\def\PY@tc##1{\textcolor[rgb]{0.73,0.13,0.13}{##1}}}
\expandafter\def\csname PY@tok@dl\endcsname{\def\PY@tc##1{\textcolor[rgb]{0.73,0.13,0.13}{##1}}}
\expandafter\def\csname PY@tok@s2\endcsname{\def\PY@tc##1{\textcolor[rgb]{0.73,0.13,0.13}{##1}}}
\expandafter\def\csname PY@tok@sh\endcsname{\def\PY@tc##1{\textcolor[rgb]{0.73,0.13,0.13}{##1}}}
\expandafter\def\csname PY@tok@s1\endcsname{\def\PY@tc##1{\textcolor[rgb]{0.73,0.13,0.13}{##1}}}
\expandafter\def\csname PY@tok@mb\endcsname{\def\PY@tc##1{\textcolor[rgb]{0.40,0.40,0.40}{##1}}}
\expandafter\def\csname PY@tok@mf\endcsname{\def\PY@tc##1{\textcolor[rgb]{0.40,0.40,0.40}{##1}}}
\expandafter\def\csname PY@tok@mh\endcsname{\def\PY@tc##1{\textcolor[rgb]{0.40,0.40,0.40}{##1}}}
\expandafter\def\csname PY@tok@mi\endcsname{\def\PY@tc##1{\textcolor[rgb]{0.40,0.40,0.40}{##1}}}
\expandafter\def\csname PY@tok@il\endcsname{\def\PY@tc##1{\textcolor[rgb]{0.40,0.40,0.40}{##1}}}
\expandafter\def\csname PY@tok@mo\endcsname{\def\PY@tc##1{\textcolor[rgb]{0.40,0.40,0.40}{##1}}}
\expandafter\def\csname PY@tok@ch\endcsname{\let\PY@it=\textit\def\PY@tc##1{\textcolor[rgb]{0.25,0.50,0.50}{##1}}}
\expandafter\def\csname PY@tok@cm\endcsname{\let\PY@it=\textit\def\PY@tc##1{\textcolor[rgb]{0.25,0.50,0.50}{##1}}}
\expandafter\def\csname PY@tok@cpf\endcsname{\let\PY@it=\textit\def\PY@tc##1{\textcolor[rgb]{0.25,0.50,0.50}{##1}}}
\expandafter\def\csname PY@tok@c1\endcsname{\let\PY@it=\textit\def\PY@tc##1{\textcolor[rgb]{0.25,0.50,0.50}{##1}}}
\expandafter\def\csname PY@tok@cs\endcsname{\let\PY@it=\textit\def\PY@tc##1{\textcolor[rgb]{0.25,0.50,0.50}{##1}}}

\def\PYZbs{\char`\\}
\def\PYZus{\char`\_}
\def\PYZob{\char`\{}
\def\PYZcb{\char`\}}
\def\PYZca{\char`\^}
\def\PYZam{\char`\&}
\def\PYZlt{\char`\<}
\def\PYZgt{\char`\>}
\def\PYZsh{\char`\#}
\def\PYZpc{\char`\%}
\def\PYZdl{\char`\$}
\def\PYZhy{\char`\-}
\def\PYZsq{\char`\'}
\def\PYZdq{\char`\"}
\def\PYZti{\char`\~}
% for compatibility with earlier versions
\def\PYZat{@}
\def\PYZlb{[}
\def\PYZrb{]}
\makeatother


    % For linebreaks inside Verbatim environment from package fancyvrb. 
    \makeatletter
        \newbox\Wrappedcontinuationbox 
        \newbox\Wrappedvisiblespacebox 
        \newcommand*\Wrappedvisiblespace {\textcolor{red}{\textvisiblespace}} 
        \newcommand*\Wrappedcontinuationsymbol {\textcolor{red}{\llap{\tiny$\m@th\hookrightarrow$}}} 
        \newcommand*\Wrappedcontinuationindent {3ex } 
        \newcommand*\Wrappedafterbreak {\kern\Wrappedcontinuationindent\copy\Wrappedcontinuationbox} 
        % Take advantage of the already applied Pygments mark-up to insert 
        % potential linebreaks for TeX processing. 
        %        {, <, #, %, $, ' and ": go to next line. 
        %        _, }, ^, &, >, - and ~: stay at end of broken line. 
        % Use of \textquotesingle for straight quote. 
        \newcommand*\Wrappedbreaksatspecials {% 
            \def\PYGZus{\discretionary{\char`\_}{\Wrappedafterbreak}{\char`\_}}% 
            \def\PYGZob{\discretionary{}{\Wrappedafterbreak\char`\{}{\char`\{}}% 
            \def\PYGZcb{\discretionary{\char`\}}{\Wrappedafterbreak}{\char`\}}}% 
            \def\PYGZca{\discretionary{\char`\^}{\Wrappedafterbreak}{\char`\^}}% 
            \def\PYGZam{\discretionary{\char`\&}{\Wrappedafterbreak}{\char`\&}}% 
            \def\PYGZlt{\discretionary{}{\Wrappedafterbreak\char`\<}{\char`\<}}% 
            \def\PYGZgt{\discretionary{\char`\>}{\Wrappedafterbreak}{\char`\>}}% 
            \def\PYGZsh{\discretionary{}{\Wrappedafterbreak\char`\#}{\char`\#}}% 
            \def\PYGZpc{\discretionary{}{\Wrappedafterbreak\char`\%}{\char`\%}}% 
            \def\PYGZdl{\discretionary{}{\Wrappedafterbreak\char`\$}{\char`\$}}% 
            \def\PYGZhy{\discretionary{\char`\-}{\Wrappedafterbreak}{\char`\-}}% 
            \def\PYGZsq{\discretionary{}{\Wrappedafterbreak\textquotesingle}{\textquotesingle}}% 
            \def\PYGZdq{\discretionary{}{\Wrappedafterbreak\char`\"}{\char`\"}}% 
            \def\PYGZti{\discretionary{\char`\~}{\Wrappedafterbreak}{\char`\~}}% 
        } 
        % Some characters . , ; ? ! / are not pygmentized. 
        % This macro makes them "active" and they will insert potential linebreaks 
        \newcommand*\Wrappedbreaksatpunct {% 
            \lccode`\~`\.\lowercase{\def~}{\discretionary{\hbox{\char`\.}}{\Wrappedafterbreak}{\hbox{\char`\.}}}% 
            \lccode`\~`\,\lowercase{\def~}{\discretionary{\hbox{\char`\,}}{\Wrappedafterbreak}{\hbox{\char`\,}}}% 
            \lccode`\~`\;\lowercase{\def~}{\discretionary{\hbox{\char`\;}}{\Wrappedafterbreak}{\hbox{\char`\;}}}% 
            \lccode`\~`\:\lowercase{\def~}{\discretionary{\hbox{\char`\:}}{\Wrappedafterbreak}{\hbox{\char`\:}}}% 
            \lccode`\~`\?\lowercase{\def~}{\discretionary{\hbox{\char`\?}}{\Wrappedafterbreak}{\hbox{\char`\?}}}% 
            \lccode`\~`\!\lowercase{\def~}{\discretionary{\hbox{\char`\!}}{\Wrappedafterbreak}{\hbox{\char`\!}}}% 
            \lccode`\~`\/\lowercase{\def~}{\discretionary{\hbox{\char`\/}}{\Wrappedafterbreak}{\hbox{\char`\/}}}% 
            \catcode`\.\active
            \catcode`\,\active 
            \catcode`\;\active
            \catcode`\:\active
            \catcode`\?\active
            \catcode`\!\active
            \catcode`\/\active 
            \lccode`\~`\~ 	
        }
    \makeatother

    \let\OriginalVerbatim=\Verbatim
    \makeatletter
    \renewcommand{\Verbatim}[1][1]{%
        %\parskip\z@skip
        \sbox\Wrappedcontinuationbox {\Wrappedcontinuationsymbol}%
        \sbox\Wrappedvisiblespacebox {\FV@SetupFont\Wrappedvisiblespace}%
        \def\FancyVerbFormatLine ##1{\hsize\linewidth
            \vtop{\raggedright\hyphenpenalty\z@\exhyphenpenalty\z@
                \doublehyphendemerits\z@\finalhyphendemerits\z@
                \strut ##1\strut}%
        }%
        % If the linebreak is at a space, the latter will be displayed as visible
        % space at end of first line, and a continuation symbol starts next line.
        % Stretch/shrink are however usually zero for typewriter font.
        \def\FV@Space {%
            \nobreak\hskip\z@ plus\fontdimen3\font minus\fontdimen4\font
            \discretionary{\copy\Wrappedvisiblespacebox}{\Wrappedafterbreak}
            {\kern\fontdimen2\font}%
        }%
        
        % Allow breaks at special characters using \PYG... macros.
        \Wrappedbreaksatspecials
        % Breaks at punctuation characters . , ; ? ! and / need catcode=\active 	
        \OriginalVerbatim[#1,codes*=\Wrappedbreaksatpunct]%
    }
    \makeatother

    % Exact colors from NB
    \definecolor{incolor}{HTML}{303F9F}
    \definecolor{outcolor}{HTML}{D84315}
    \definecolor{cellborder}{HTML}{CFCFCF}
    \definecolor{cellbackground}{HTML}{F7F7F7}
    
    % prompt
    \makeatletter
    \newcommand{\boxspacing}{\kern\kvtcb@left@rule\kern\kvtcb@boxsep}
    \makeatother
    \newcommand{\prompt}[4]{
        \ttfamily\llap{{\color{#2}[#3]:\hspace{3pt}#4}}\vspace{-\baselineskip}
    }
    

    
    % Prevent overflowing lines due to hard-to-break entities
    \sloppy 
    % Setup hyperref package
    \hypersetup{
      breaklinks=true,  % so long urls are correctly broken across lines
      colorlinks=true,
      urlcolor=urlcolor,
      linkcolor=linkcolor,
      citecolor=citecolor,
      }
    % Slightly bigger margins than the latex defaults
    
    \geometry{verbose,tmargin=1in,bmargin=1in,lmargin=1in,rmargin=1in}
    
   

\begin{document}
    
    \begin{center}
    
    \LARGE
    \hypertarget{adv-help-mailbox-analysis}{%
\section*{ADV Help Mailbox Analysis}\label{adv-help-mailbox-analysis}}

\normalsize
Fall 2020

Author: Ethan Creagar

\end{center}

\tableofcontents

    \hypertarget{description}{%
\subsubsection{Description:}\label{description}}

This project, authorized and supervized by Nate Williams and Colorado
State University Advancement, explored the history of the ADV Help
Outlook inbox from 2016-2020 to search for insight on trends and provide
baseline numbers for the helpdesk. Specifically, the following points
will be addressed:

\begin{center}\rule{0.5\linewidth}{0.5pt}\end{center}

\begin{enumerate}
\def\labelenumi{\arabic{enumi})}
\item
  How many tickets do we recieve per month? How many do we solve on
  average?
\item
  What is the number of incoming tickets per week? Per day? When are our
  times of increased and decreased volume?
\item
  What is the category breakdown on incoming tickets? How has this
  changed over time?
\item
  What is the breakdown of tickets completed by category? By technician?
  By time period?
\end{enumerate}

\begin{center}\rule{0.5\linewidth}{0.5pt}\end{center}

These questions give us insight into the workings of the helpdesk and
help us set goals and be better prepared for the issues of CSUA
employees in the future.


    \hypertarget{part-1---inbox-basics}{%
\section{Part 1 - Inbox Basics}\label{part-1---inbox-basics}}

In Part 1, we look to answer some basic questions about the inbox, such
as the quantity of tickets recieved and how this has changed over time.

\hypertarget{a-how-many-tickets-have-we-recieved}{%
\subsubsection{a: How many tickets have we
recieved?}\label{a-how-many-tickets-have-we-recieved}}


    \begin{Verbatim}[commandchars=\\\{\}]
Total tickets: 15377
    \end{Verbatim}

    Through the last 5 years, ADV Help has recieved 15,377 emails.

    \hypertarget{b-on-average-how-many-tickets-do-we-solve}{%
\subsubsection{b: On average, how many tickets do we
solve?}\label{b-on-average-how-many-tickets-do-we-solve}}

    \begin{Verbatim}[commandchars=\\\{\}]
Completed 5660 out of 6008 unique tickets: 94.21\% (plus/minus 5\%)
    \end{Verbatim}

    By indexing the data to only include the tickets whose categories
contain the word ``Done'', we can see how many tickets we have
completed. We can compare this against only unique emails, defined as
emails without ``RE:'' at the beginning of them as this would indicate
that the email was part of a chain, and find the percentage of tickets
we've completed. This percentage relies on accurate ticketing of data,
so I built in a 5\% reduction and Confidence Interval of 5\% to adjust
for marking the same ticket as ``Done'' twice at different points in its
chain.

    \hypertarget{c-how-have-these-trends-changed-over-time}{%
\subsubsection{c: How have these trends changed over
time?}\label{c-how-have-these-trends-changed-over-time}}


    \begin{center}
    \adjustimage{max size={0.9\linewidth}{0.9\paperheight}}{output_13_0.png}
    \end{center}
    { \hspace*{\fill} \\}
    
    Here we see the trend of our emails recieved from February of 2016 to
April of 2020. We see that there is an upward trend in this data and we
can calculate this trend below.

    \begin{Verbatim}[commandchars=\\\{\}]
Intercept: 88.303
Beta 1: 5.226
    \end{Verbatim}

    The line of best fit through the data can be quantified as
\(\hat{y} = 88.303 + 5.226\hat{x}\). \(\beta_1\) is likely affected by
the outlier of March 2020.

\textbf{Over time, the number of emails recieved by ADV Help has grown
at a rate of about 5.23 emails per month, or about 63 emails per year.}

    \hypertarget{part-2}{%
\section{Part 2:}\label{part-2}}

In part 2, we want to examine how many tickets we get in different time
periods, whether this be morning vs.~afternoon, months, years, or a
combination of these things.

\hypertarget{how-can-we-allocate-staff-to-match-our-times-of-highest-volume-of-tickets}{%
\subsubsection{How can we allocate staff to match our times of highest
volume of
tickets?}\label{how-can-we-allocate-staff-to-match-our-times-of-highest-volume-of-tickets}}



    \textbf{Methodology}

We can plot emails recieved by time frame with many options, including:

\begin{itemize}
\tightlist
\item
  ``Date'': Which weekday - Monday to Sunday - the email was recieved
\item
  ``AMPM'': If the email was recieved in the morning or afternoon
\item
  ``Hour'': Which hour of the day - 1 to 23 - the email was recieved
\item
  ``Minute'': Which minute - 1 to 60 - of the hour the email was
  recieved
\item
  ``Month'': Which month - 1 to 12 - the email was recieved
\item
  ``Day'': Which day - 1 to 31 - the email was recieved
\item
  ``Year'': Which year the email was recieved in
\end{itemize}

With this information, we can plot something simple like whether we
recieve more tickets in the morning or afternoon:

    \begin{center}
    \adjustimage{max size={0.9\linewidth}{0.9\paperheight}}{output_22_0.png}
    \end{center}
    { \hspace*{\fill} \\}
    
    and see that in our history, we have recieved more emails after 12:00 PM
than before. This could be helpful, but a more in depth version might
include weekday in it as well to be more thorough. Including weekday,
for example, would look like this:

    \begin{center}
    \adjustimage{max size={0.9\linewidth}{0.9\paperheight}}{output_25_0.png}
    \end{center}
    { \hspace*{\fill} \\}
    
    Here we can see that amount of tickets recieved in the afternoon is
higher earlier in the week and trends downward as the week goes on.
Tuesdays and Wednesdays are our busiest days, while Fridays see the
least emails of any weekday. Friday mornings also tend to be busier than
friday afternoons, the only day where this is true. Mondays however show
the lagest difference between afternoon and morning.

    An even more in depth version might ask whether this trend has changed
over the last few years. Including year in the plot would look like
this:


    \begin{center}
    \adjustimage{max size={0.9\linewidth}{0.9\paperheight}}{output_29_0.png}
    \end{center}
    { \hspace*{\fill} \\}
    
    We can see that although in the past we've recieved more emails in the
afternoon, when we plot this for each year the trend shows that this
trend is not true for 2020 so far and that we are recieving more emails
in the morning this year.

There are many interesting combinations that can give us trends of when
we should look out for more emails.

    \begin{center}
    \adjustimage{max size={0.9\linewidth}{0.9\paperheight}}{output_31_0.png}
    \end{center}
    { \hspace*{\fill} \\}
    
    Here, we see that March is our largest month for emails recieved.
However, this is misleading, as we see in the plot below.

    \begin{center}
    \adjustimage{max size={0.9\linewidth}{0.9\paperheight}}{output_33_0.png}
    \end{center}
    { \hspace*{\fill} \\}
    
    By organizing the variables this way, we can see trends by year
colorized by month. This allows us to see if there are any trends with
certain months being more or less busy. If we had just broken this down
by month, the data shows March as one of our busiest months, but looking
at it this way we can see that apart from the outlier in 2020 the
opposite is usually true and March is not generally a large month for
us.


    \begin{center}
    \adjustimage{max size={0.9\linewidth}{0.9\paperheight}}{output_35_0.png}
    \end{center}
    { \hspace*{\fill} \\}
    
    This plot shows us the inverse of the last plot - each month, colorized
by which year the ticket was recieved in. This gives us insight to be
able to compare the first few months of a year to previous years or
specific months against previous years. We can see here that most months
seem to be rising over time, which agrees with our previous insight that
we are recieving more and more emails. We can also see that even before
lockdown, January and February were shaping up to be as busy or busier
than last year, meaning not all of the influx of emails should be
attributed to working from home and that we were likely to continue to
see growth in the inbox regardless.


    \begin{center}
    \adjustimage{max size={0.9\linewidth}{0.9\paperheight}}{output_37_0.png}
    \end{center}
    { \hspace*{\fill} \\}
    
    Here we break the last two down further by splitting the groups into
Year, morning and afternoon. We can see use this plot to see whether
changes over time are consistant between morning and afternoon. We can
see that in some months, like October, we have been historically more
likely to recieve emails in the afternoon.

    More examples of plots with insights are shown below.

   
    \begin{center}
    \adjustimage{max size={0.9\linewidth}{0.9\paperheight}}{output_40_0.png}
    \end{center}
    { \hspace*{\fill} \\}
    
    \begin{itemize}
\tightlist
\item
  \textbf{Total Emails - Hour:} We reviece the most emails between 9 and
  10 o'clock, followed by between 10 and 11 o'clock and between 2 and 3
  o'clock.
\item
  \textbf{Total Emails - Hour and Weekday:} The amount of emails we
  recieve by hour is fairly constant across the day of the week. We see
  more emails between 9:00 and 12:00 on Tuesdays than any other day,
  while we see the most emails between 2:00 and 4:00 on Wednesdays.
\item
  \textbf{Total Emails - Morning vs.~Afternoon by Day of the Month:} We
  see a similar breakdown to the plot below here, but we can see that we
  get more emails in the afternoon later in the month than we do earlier
  in the month.
\item
  \textbf{Total Emails - Day of the Month:} We see the most emails on
  the 5th and 6th of the month, followed by the 30th. We recieve
  slightly less emails later in the month than earlier in the month, but
  it's not a very significant difference.
\item
  \textbf{Total Emails - Year:} We have looked at this in different
  ways, but this plot shows us that this year will be our highest
  emailed year by far.
\item
  \textbf{Total Emails = Morning vs.~Afternooon by Month:} we see more
  email in the afternoon in the later months than we do in the first 6
  months.
\end{itemize}

    \begin{center}\rule{0.5\linewidth}{0.5pt}\end{center}

There are many more combinations of plots we could create, but the
insight drawn from these is valuable.

To answer our question, we can summarize some \textbf{trends of the
times of increased and decreased volume:}

\begin{itemize}
\item
  \textbf{Afternoons} are typically more busy than mornings. We have
  recieved almost 53\% of our tickets in the afternoon, although that
  trend has changed this calendar year. In particular, we see our most
  emails between 9:00 and 11:00 AM and 1:00 and 3:00 PM
\item
  \textbf{Tuesdays and Wednesdays} are our busiest days of the week.
  Monday afternoons are busy, while mornings have much less volume.
  Friday is our least busy time, both morning and afternoon.
\item
  \textbf{Earlier in the Month} tends to be slightly busier than later
  in the Month.
\item
  \textbf{The Month of the year} in which we recieved the most email
  does not have much of a constant trend. Generally, \textbf{October,
  August and February} are times of more emails, while \textbf{December}
  is our least-emailed month.
\end{itemize}

    \hypertarget{part-3}{%
\section{Part 3}\label{part-3}}

In this section, we break our inbox down into categories to see if we
can find any trends based on the different categories. We look to
explore this by different time period like we did in part 2.

\hypertarget{a-what-is-the-category-breakdown-on-incoming-tickets}{%
\subsection{a: What is the category breakdown on incoming
tickets?}\label{a-what-is-the-category-breakdown-on-incoming-tickets}}

Using our function, we can plot Categories by time or against one
another by time. For example, below is a plot of Email category tickets
vs Hardware category tickets over the last 4 years.


    \begin{center}
    \adjustimage{max size={0.9\linewidth}{0.9\paperheight}}{output_51_0.png}
    \end{center}
    { \hspace*{\fill} \\}
    
    
    \begin{center}
    \adjustimage{max size={0.9\linewidth}{0.9\paperheight}}{output_52_0.png}
    \end{center}
    { \hspace*{\fill} \\}
    
    The rest of the categories are plotted above. We can see that most of
them have a spike recently and are generally trending upward. Most of
these spikes are easily explainable - for instance, the \textbf{Remote}
category spikes around March of 2020, when the division was starting to
work from home and needed our remote connection guides and software.
\textbf{Adobe} has a spike in November/December of 2019, the month when
the technicians were upgrading all machines to the new Adobe Acrobat
2017.


    The breakdown of total emails by subject can be seen below.


            \begin{tcolorbox}[breakable, size=fbox, boxrule=.5pt, pad at break*=1mm, opacityfill=0]
\prompt{Out}{outcolor}{30}{\boxspacing}
\begin{Verbatim}[commandchars=\\\{\}]
                  From  Total  Percentage
Email Categories
                  5601  15251        36.7
Adobe              464  15251         3.0
Email             1264  15251         8.3
Hardware          3238  15251        21.2
Network           1324  15251         8.7
Personnel          488  15251         3.2
Phone             1124  15251         7.4
Purchasing         123  15251         0.8
Remote             624  15251         4.1
Software          1001  15251         6.6
\end{Verbatim}
\end{tcolorbox}
        
    \hypertarget{b-what-is-does-this-breakdown-look-like-by-time-period}{%
\subsection{b: What is does this breakdown look like by time
period?}\label{b-what-is-does-this-breakdown-look-like-by-time-period}}

We might want to know if there are any trends of categoriezed tickets to
know if we see more errors or slowness at any times in specific.

 
    Our Group Categories and Plot Categories functions can give us the same
insight as we recieved above, but this time broken down into categories
so we can see when we are recieving the most of a certain type of
category in a more detailed way than just over each month like above.

    \begin{center}
    \adjustimage{max size={0.9\linewidth}{0.9\paperheight}}{output_61_0.png}
    \end{center}
    { \hspace*{\fill} \\}
    
    For example, here we see the \textbf{Hardware} category broken down by
month. Again, this may not be as insightful as we'd like and we might
have a bit of a bias toward March and April due to the large spike of
emails in that time period of 2020. We can break this down further to
get a better idea of this category's time distribution.


    \begin{center}
    \adjustimage{max size={0.9\linewidth}{0.9\paperheight}}{output_63_0.png}
    \end{center}
    { \hspace*{\fill} \\}
    
    We see here that we are correct - There was a big spike in February and
March of 2020. We could, however, argue whether the February spike was
out of the ordinary based on the fact that \textbf{we seem to see many
Hardware tickets in February each year. We also seem to see spikes in
October/November}.


    \begin{center}
    \adjustimage{max size={0.9\linewidth}{0.9\paperheight}}{output_65_0.png}
    \end{center}
    { \hspace*{\fill} \\}
    
    The first plot breaks down each year by month so we can see trends by
month inside each year, while the second breaks down each month by year
to see how our number of emails recieved in each month have grown or
shrunk by year. This gives us the insight that \textbf{like emails in
general and most categories of emails, Hardware tickets are rising farly
consistantly by year}.


    \begin{center}
    \adjustimage{max size={0.9\linewidth}{0.9\paperheight}}{output_67_0.png}
    \end{center}
    { \hspace*{\fill} \\}
    
    This plot tells us that \textbf{certain months, such as October and
November, and have generally had a larger discrepency between the
morning and evening emails.}

    So to answer the question ``does Outlook run slower on Monday
mornings?'', we can do the following:


    \begin{center}
    \adjustimage{max size={0.9\linewidth}{0.9\paperheight}}{output_70_0.png}
    \end{center}
    { \hspace*{\fill} \\}
    

    \begin{center}
    \adjustimage{max size={0.9\linewidth}{0.9\paperheight}}{output_71_0.png}
    \end{center}
    { \hspace*{\fill} \\}
    
    and conclude that actually, we get email tickets at almost the exact
same distribution as we get all tickets, with a slight tilt toward
Wednesday afternoons. (Keep in mind that we can only make an inference
as far as the ticketing system goes. If someone doesn't email us to tell
us that their Outlook is slow, that trend obviously won't show in the
plots.)

However, with remote tickets:

    \begin{center}
    \adjustimage{max size={0.9\linewidth}{0.9\paperheight}}{output_73_0.png}
    \end{center}
    { \hspace*{\fill} \\}
    
    We see that we get a lot more tickets in the mornings than we do in the
afternoons, which could help us be more on the lookout for these tickets
in the mornings. We can see that this trend is also heavily influenced
by 2020.

    \begin{center}
    \adjustimage{max size={0.9\linewidth}{0.9\paperheight}}{output_75_0.png}
    \end{center}
    { \hspace*{\fill} \\}
    

    \begin{center}
    \adjustimage{max size={0.9\linewidth}{0.9\paperheight}}{output_76_0.png}
    \end{center}
    { \hspace*{\fill} \\}
    
    Going through the plots and searching for interesting and/or unexpected
trends yeilded the insights explored in the plots below.


    \begin{center}
    \adjustimage{max size={0.9\linewidth}{0.9\paperheight}}{output_78_0.png}
    \end{center}
    { \hspace*{\fill} \\}
    
    \begin{itemize}
\item
  \textbf{Software Tickets}: Plotting software tickets vs total tickets
  by year recieved, we can see that \textbf{software tickets}, tickets
  including keywords such as ``download'', ``update'', ``software'',
  ``installation'', and ``licensing'', \textbf{seem to be growing at a
  faster rate than our tickets overall}. As a total percentage of our
  tickets, software tickets are also growing year-by-year with a spike
  in 2020.
\item
  \textbf{Personnel tickets}: \textbf{We see that we get a spike in
  personnel tickets,} tickets including keywords such as ``Hire'',
  ``On-boarding'', ``Welcome'', and ``Resignation'', \textbf{in August}.
  Looking at the plot of Personnel tickets by year broken into month
  recieved, we can wee that \textbf{this spike in August/September is
  farily consistant over at least the last three years,} so this spike
  is fairly reliable.
\end{itemize}

    \begin{center}
    \adjustimage{max size={0.9\linewidth}{0.9\paperheight}}{output_80_0.png}
    \end{center}
    { \hspace*{\fill} \\}
    
    \begin{itemize}
\item
  \textbf{We see the most network tickets on Tuesday mornings}. It is
  not out of the ordinary to see an increase in tickets on Tuesdays, but
  we know that mornings are generally less busy than afternoons.
\item
  \textbf{Email tickets by month recieved:} By breaking down our
  ``email'' tickets, tickets with keywords like ``Outlook'',
  ``Mailbox'', and ``Email'', we can see that we see our largest spike
  in January. This may be an interesting trend to look into as it
  counters our findings above that January is generally not a time of
  large email volume.
\item
  \textbf{Purchasing tickets by month recieved}: Our purchasing tickets,
  tickets with keywords such as ``order'', ``purchase'',
  ``replacement'', or ``delivery'', sees by far its largest spike in
  September.
\end{itemize}

    \begin{center}\rule{0.5\linewidth}{0.5pt}\end{center}

    \hypertarget{part-4}{%
\section{Part 4:}\label{part-4}}

In part 4, we want to answer questions about the breakdown of our total
tickets in each category. We can use this information to inform
descisions about what tickets to devote the most resources to, how we
should train employees, and more.

\hypertarget{a-what-is-the-breakdown-of-tickets-completed-by-category}{%
\subsection{a: What is the breakdown of tickets completed by
category?}\label{a-what-is-the-breakdown-of-tickets-completed-by-category}}

    We now have a breakdown of what tickets we recieve the most of and when,
which can be useful to know what kind of tickets we should devote the
most resources to. It might also be interesting to see what kind of
tickets we complete at the highest rate to see if we struggle in some
categories more than others. Note that these percentages lean on the
side of too high due to tickets marked Done twice.


    \begin{Verbatim}[commandchars=\\\{\}]
Completed Tickets:
                   From  Total  Percentage
Email Categories
                  2277   6766        33.7
Adobe              223   6766         3.3
Email              632   6766         9.3
Hardware          1443   6766        21.3
Network            657   6766         9.7
Personnel          274   6766         4.0
Phone              568   6766         8.4
Purchasing          58   6766         0.9
Remote             238   6766         3.5
Software           396   6766         5.9



Total Tickets:
                   From  Total  Percentage
Email Categories
                  5601  15251        36.7
Adobe              464  15251         3.0
Email             1264  15251         8.3
Hardware          3238  15251        21.2
Network           1324  15251         8.7
Personnel          488  15251         3.2
Phone             1124  15251         7.4
Purchasing         123  15251         0.8
Remote             624  15251         4.1
Software          1001  15251         6.6
    \end{Verbatim}

    We add some noise to the Completed tickets group to account for the
lower percentage of Uncategorized tickets and plot the two data frames
side-by-side.

    \begin{center}
    \adjustimage{max size={0.9\linewidth}{0.9\paperheight}}{output_90_0.png}
    \end{center}
    { \hspace*{\fill} \\}
    
    We can see that our numbers follow pretty closely, and with the
inaccuracy of the done tickets we can say that most of the percentages
are likely very close to each other without duplicates. The interesting
columns then are the ones that have lower or very close percentages in
the Done data frame. These are \textbf{Hardware, Remote, and Software.}
If our data is accurate, this would mean that \textbf{these are the
tickets that go uncompleted most often}. This could be something to look
out for in the future.

    \hypertarget{b-by-technician}{%
\subsection{b: By technician?}\label{b-by-technician}}

    Dividing the tickets by technician, we can compare what kind of tickets
a technician responds to most vs what kind of tickets we recieve the
most to see if our technicians have any obvious gaps in the type of
tickets they respond to. For instance, here are the number of tickets
I've completed (those marked as Done - Ethan) vs the total numbers.


    \begin{Verbatim}[commandchars=\\\{\}]
Tickets completed by Ethan:
    Email Categories  From  Total  Percentage
29                    271    944        28.7
30            Adobe    74    944         7.8
31            Email    90    944         9.5
32         Hardware   217    944        23.0
33          Network    70    944         7.4
34        Personnel    34    944         3.6
35            Phone    53    944         5.6
36       Purchasing    16    944         1.7
37           Remote    65    944         6.9
38         Software    54    944         5.7


Total Tickets completed
                   From  Total  Percentage
Email Categories
                  5601  15251        36.7
Adobe              464  15251         3.0
Email             1264  15251         8.3
Hardware          3238  15251        21.2
Network           1324  15251         8.7
Personnel          488  15251         3.2
Phone             1124  15251         7.4
Purchasing         123  15251         0.8
Remote             624  15251         4.1
Software          1001  15251         6.6
    \end{Verbatim}

    We can see that my responses (on top) have a bit of an advantage in the
Adobe and Email categories, which means I respond to tickets of those
sort at a higher rate than our historical average. However, I respond at
a lower percentage to Network, Phone, and Software category tickets,
meaning I may have a bit of work to do in terms of finishing those
tickets.

    We can gather this data for any technician whose tickets are still
marked ``Done - Technician'' in our inbox. For example, here we can see
Nate's tickets:

    \begin{Verbatim}[commandchars=\\\{\}]
Tickets Completed by Nate:
                   From  Total  Percentage
Email Categories
                   896   2630        34.1
Adobe               62   2630         2.4
Email              236   2630         9.0
Hardware           511   2630        19.4
Network            282   2630        10.7
Personnel          141   2630         5.4
Phone              269   2630        10.2
Purchasing          13   2630         0.5
Remote              74   2630         2.8
Software           146   2630         5.6


Total Tickets completed:
                   From  Total  Percentage
Email Categories
                  5601  15251        36.7
Adobe              464  15251         3.0
Email             1264  15251         8.3
Hardware          3238  15251        21.2
Network           1324  15251         8.7
Personnel          488  15251         3.2
Phone             1124  15251         7.4
Purchasing         123  15251         0.8
Remote             624  15251         4.1
Software          1001  15251         6.6
    \end{Verbatim}

    Comparing the emails from Done - Nate to the total ticket breakdown, we
can see that his responses follow the distribution pretty closely. This
would make sense, since he's solved about 1/6 of the total ADV Help
tickets. Nate's reponses also lean toward Network tickets slightly,
which we would expect with what kind of tickets he tends to take care
of.

    \textbf{Training} is one of the areas that helpdesks might look to
improve upon. We can use the style of data frame above to gather
information about how effectively technicians have been trained, and
which areas of training may still need to be improved upon. Below are
three Technicians. Agent 1 worked in the Helpdesk from February of 2016
to May of 2019. Agent 2 has worked in the Helpdesk from October of 2018
to present, and Agent 3 has worked in the Helpdesk from October 2019 to
present.

    \begin{Verbatim}[commandchars=\\\{\}]
Agent1:
                   From  Total  Percentage
Email Categories
                   512   1493        34.3
Adobe               32   1493         2.1
Email              148   1493         9.9
Hardware           334   1493        22.4
Network            159   1493        10.6
Personnel           55   1493         3.7
Phone              138   1493         9.2
Purchasing          19   1493         1.3
Remote              39   1493         2.6
Software            57   1493         3.8

Agent2:
                   From  Total  Percentage
Email Categories
                   271    944        28.7
Adobe               74    944         7.8
Email               90    944         9.5
Hardware           217    944        23.0
Network             70    944         7.4
Personnel           34    944         3.6
Phone               53    944         5.6
Purchasing          16    944         1.7
Remote              65    944         6.9
Software            54    944         5.7

Agent3:
                   From  Total  Percentage
Email Categories
                   151    574        26.3
Adobe               26    574         4.5
Email               57    574         9.9
Hardware           148    574        25.8
Network             47    574         8.2
Personnel           15    574         2.6
Phone               29    574         5.1
Purchasing           4    574         0.7
Remote              29    574         5.1
Software            68    574        11.8
    \end{Verbatim}


    \begin{center}
    \adjustimage{max size={0.9\linewidth}{0.9\paperheight}}{output_103_0.png}
    \end{center}
    { \hspace*{\fill} \\}
    
    Due to these time differences, this can give us insights about training.
For instance, we see that the percentage of Network tickets, tickets
with keyworks such as ``internet'', ``wifi'', ``domain'',
``permissions'', ``drives'', and ``network drives'', rises from Agent 3
and 2 to Agent 1. this shows us that with more time, agents may get more
comfortable with network tickets or be asked to answer more of them. We
see a similar trend with Phone tickets. On the opposite, we see a
category such as hardware tickets, tickets including ``station'',
``setup'', and other computer and printer hardware words. We can see
that this category gets smaller as the agents answer more tickets, this
could show us that as agents are further trained, they are called upon
less to answer tickets dealing with hardware.

Overall, the categories are fairly even, showing us that although
helpdesks may get more comfortable with certain types of tickets over
time, \textbf{training is doing a good job of preparing new hires to
deal with all types of tickets.}

\begin{center}\rule{0.5\linewidth}{0.5pt}\end{center}

To conclude our question about completed tickets broken down by
category, we can say that we tend to complete tickets at about the same
rate as we recieve them in each category. Our biggest discrepencies are
in the Hardware, Software and Remote categories. We can also make
conclusions about our quality of training based on this breakdown. In
particular, we can say, with a small sample size, that newer hires tend
to complete tickets at about the same rate across categories that more
experienced technicians do. We do see an increased amount of Hardware
tickets and software tickets completed by newer hires than older hires,
and more network tickets completed by older hires.

    \hypertarget{conclusion}{%
\section{Conclusion}\label{conclusion}}

    \hypertarget{inbox-basics}{%
\subsubsection{Inbox basics:}\label{inbox-basics}}

\begin{itemize}
\tightlist
\item
  \textbf{ADV Help has recieved 15,377 emails over the last 5 years.}
  This is a trend that has been growing - on average we recieve about 5
  more emails every month than the last. We solve about 95 percent of
  tickets (\(\pm5 \%\)).
\end{itemize}

\hypertarget{when-do-we-recieve-the-most-tickets}{%
\subsubsection{When do we recieve the most
tickets?:}\label{when-do-we-recieve-the-most-tickets}}

\begin{itemize}
\item
  \textbf{Afternoons} are typically more busy than mornings. We have
  recieved almost 53\% of our tickets in the afternoon, although that
  trend has changed this calendar year. In particular, we see our most
  emails between 9:00 and 11:00 AM and 1:00 and 3:00 PM
\item
  \textbf{Tuesdays and Wednesdays} are our busiest days of the week.
  Monday afternoons are busy, while mornings have much less volume.
  Friday is our least busy time, both morning and afternoon.
\item
  Generally, \textbf{October, August and February} are times of more
  emails, while \textbf{December} is our least-emailed month.
\end{itemize}

\hypertarget{what-does-our-inbox-breakdown-look-like-by-ticket-category}{%
\subsubsection{What does our inbox breakdown look like by ticket
category?:}\label{what-does-our-inbox-breakdown-look-like-by-ticket-category}}

\begin{itemize}
\item
  \textbf{About 20 percent of the total tickets we recieve are Hardware
  tickets, our highest category.} Email and Network tickets are next,
  with just over 8 percent each, then Software tickets.
\item
  \textbf{As opposed to out total tickets, we recieve more Remote
  tickets in the morning}. Other categories do not see this same trend.
\item
  \textbf{Software tickets are growing at a faster rate than total
  tickets are.} We see that year by year, a higher percentage of our
  total tickets are software based.
\item
  We tend to see a \textbf{spike in personnel tickets in August.} This
  has been a consistent trend accross the last few years.
\item
  \textbf{We see the most email tickets in January.} This is a
  significant finding because January is generally one of our lowest
  volume times for lowest tickets. We may want to examine whether there
  is a reason this would be the case.
\item
  \textbf{Purchasing tickets see a spike in September}. We should be on
  the lookout for more tickets involving purchasing, hardware
  replacements, etc. at this time of the year.
\item
  \textbf{We see the most network tickets on Tuesday mornings}. It is
  not out of the ordinary to see an increase in tickets on Tuesdays, but
  we know that mornings are generally less busy than afternoons. We also
  don't see a consistancy in increased Network tickets in the mornings,
  which could lead us to believe that Tuesday morning Network issues may
  need to be examined further. This trend is most pronounced in 2020.
\end{itemize}

\hypertarget{what-does-our-breakdown-of-tickets-completed-look-like-by-category}{%
\subsubsection{What does our breakdown of tickets completed look like by
category?:}\label{what-does-our-breakdown-of-tickets-completed-look-like-by-category}}

\begin{itemize}
\item
  \textbf{We generally complete tickets at a similar rate to which we
  recieve them in each cateogry}. The Hardware, Remote, and Software
  ticket categories have the biggest discrepencies in them between
  recieved and completed.
\item
  \textbf{New hires do not have much of a difference in tickets
  completed by category than more experienced employees do.} We do see
  that (with a small sample size) new tickets complete a higher
  percentage of hardware and software tickets, whereas more experienced
  technicians complete more network tickets.
\end{itemize}



    % Add a bibliography block to the postdoc
    
    
    
\end{document}
